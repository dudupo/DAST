\documentclass{article}


\usepackage[utf8]{inputenc}
\usepackage[a4paper, total={6in, 9in}]{geometry}
\usepackage{braket}
\usepackage{xcolor}
\usepackage{amsmath}
\usepackage{amsfonts}
\usepackage{tikz}
\usepackage{svg}
\usepackage{graphicx}
\usepackage{media9}
\usepackage{float}
\usetikzlibrary{calc}
\usepackage{array}
\usepackage[ruled,vlined,linesnumbered]{algorithm2e}

\usepackage[
backend=biber,
style=alphabetic,
sorting=ynt
]{biblatex}



\newcommand{\commentt}[1]{\textcolor{blue}{ \textbf{[COMMENT]} #1}}
\newcommand{\ctt}[1]{\commentt{#1}}
\newcommand{\prb}[1]{ \mathbf{Pr} \left[ {#1} \right]}
\newcommand{\onotation}[1]{\(\mathcal{O} \left( {#1}  \right) \)}
\newcommand{\ona}[1]{\onotation{#1}}
\newcommand{\nvar}[2]{ \( #1_{1}, #1_{2} ... #1_{#2}  \) }
%\newenvironment{proof}[0]{\paragraph{Proof.}}{}
%\newenvironment{remark}[0]{\textit{remark}}{}
\newenvironment{cor}[0]{\paragraph{Corollary.}}{}
\newenvironment{example}[0]{\paragraph{Example.}}{}
%\newenvironment{thm}[0]{\paragraph{Theorem.}}{}
\newtheorem{prop}{Proposition}
\newtheorem{ex}{Exercise}
\newtheorem{sol}{Solution}
\newtheorem{theorem}{Theorem} 		
\newtheorem{thm}{Theorem}[section]
\newtheorem{conj}[thm]{Conjecture} 	
\newtheorem{lemma}[thm]{Lemma}
\newtheorem{corollary}[thm]{Corollary} 
\newtheorem{claim}[thm]{Claim}
\newtheorem{proposition}[thm]{Proposition}
\newtheorem{definition}{Definition} 
\newtheorem{remark}{Remark}
 


% \addbibresource{sample.bib} %Import the bibliography file
\begin{document}    
\maketitle

\title{<++>}

Recitation Week 2 - Complexity and Correctness

Abstract

In this course (and many more), we will encounter algorithm. As
you've scen in Intro, a single task can have many different solutions and
implementations. One of the things that differ between sohstiona is their
running time, or time complexity. We contime here the dicumsion
you had in clas om this subject. We alas add the notion of algorithm's
correctness, as we woul! ike to analyse only algorithms that solve the
given problem for any given input, and show how we can prove this halds
for a specific algorithm.

 

1 Asymptotic notations - O,2,0

Definition 1.1. (Reminder) Let f,y:N—+ B+, We define:
* fit) =Ofin)) == 30 >0 AngEN Yn>ne fin) < Cyn)
@ fin} =A2igin)) <= Fe>0 3noEN Yo> no eg(n) < fin)

© fa) = Sf9ir}) fl) = Oia) A Fin}

 

 

AUein))

Draw schematic examples. For @ preferably of a function that “goes muta”
betuwet tua lines, for example.

In exence, Big O notation implies that “eventually, g ® larger that 7.
Shuilarly, Big 8 notation implies that “eventually, g ® smaller than (°. Another
way to put it ® that asymptotically, g is an upper bowl for f [for O, lower
bound for 9). For ©, it describes asymptotic similarity, in the seuse that f
“behaves like” g for lange values of n. We will se some formalization of th
idea later talay.

 

 

1.1 Properties
General properties

Claim 1.2. (Reflerive)
For any f. fin) = O(n)

Pwof. Choose ng =1,C = 1, for any n > ng we have f(n) < fin). a

 

We sad last week this also hokls for 6. Does it also bold for 27 (was noted
dn elas

 
Claim 1.3. (Aatiaymmetr:
For any fg, § =Olg) fg = 2(f).

Remark 1.4. This is not exactly the came as an antisynsnuetric relation as you
saw in Discrete Math! f = Ofg) and g= O(f) does not imply f = 9.

Prof. f = O(g) > g = Of): Let ng, C withn > ng => fin) < Cy(n). Choose

e= A, thus n > ng =Sefin) < a(n)
QU) => fF = Oly): Similarly. oO

 

  

Claim 1.5. Symmetric f =O(g) —> 9 = Off)
Prof.

f= (9) <= (f=0@AT= 20) <= (9=2U)A9= 00) <> 9 = O/)
a
Exerelse 1.6. Show that the rtations O, 2,0 are transitive.

(Transitiaity of O appears in Birt)
Remark 1.7. This justifies thinking about 6 as an equivalence relation. In the
sane way - f = Ofy) means “y is larger than f7: This is an order relation fon
equivalence clases).

Algebraic properties

Question 1.8. (Can do this using poll)

Let fig: NB.

Prove or d&prove the following claims:

1. diate rad = 0 => gin) = Q6fin))
2 #6) = 0(F(3)}

3. f(a) = O(n) = Pm) =O(™))
4. fn) = Ofgin}) = 26 = (20)

Solution. (With possilsle examples)
‘ug = (u)8 ‘a= (ul f - apdexg wsyeg “p

0 = FF = 140 =F (OO = F myep peronas v jo veg ona, “¢
uf = (ul f - aplmexg weg 7
"| SSIOe Tt] Toppenb oy rpg com, “y

Example 1.9. (Factorial)

Prove nl = Ofn")
Prof. on! =[TL, i <TTL, n= 2" a
We will see another proof, just to work our iduction muscles:
Prof. (wing iduction) Choosing © = 1 and ne = U, prove by iluction on
that Yn > ng inl < Cn".
Base: for n = 1 this is immediate.

Assumption: Assume this hokls for » — 1, that is (m —1)) <(m—1)"-*.
Step: For n, we have:

mls n-fn— 1)! <nfa— iyo! < nent) =n”
And Bob's your uncle. a

Exercise 110. (Appears in Exercise 1) If f(n) = Olg(n)) ond limo f(n) =
ox, then log(f(n)) = Ollog(a(n)).

Corollary 1.11. log(n!) = Ofmleg(n))

Proof. We showed that mn! = Ofn"), andn! ——+ oo, thus lbg{n!) = Oflogin™)) =
Ofmlog(n)). a
2 Running time of algorithms

In computational tasks the difficulty lies in solving the given problem for lange

distances. The difficulty usually grows monotonically with the input size, so we

would like to analyse it, meaning to calculate the amount of resources used by it

asa function of the input size. To address this question we must define what

resources we nreasure /oount and the way in which we nueasure/oount then.
Given an algorithm, we woukl like to know: what is the algorithm's

@ Space complexity?
« Best time complexity?

@ Worst time complexity?

By “space” we mean memory. Both memory amd tine are machine-depemlent
axl a8 you saw in class, we would like to study algorithms, the abstraction of
programs, that do not depeml on the omling-language or the machine on which
the instructions might run.

Preparatory Question. The asymptotic notations (O,8,@) are equivalence
classes that “ignore” two paraneters of functions:

1. behaviour on finite prefixes of inputs
2. constant scalings

As was said in class, thik allows us to ignore machine-specific properties. What
sort of such properties can we ignomw this way? Can you think of an example
showlng the necessity of this approach?

An example to show the necessity of the first property being ignored can be
algorithms with a constant-time pre-processing stage, like constructing several
dictionaries in python. For small inputs this stage sem to nuke the algorithn
inefficient, but its advantages become clear in langer inputs.

An example for the secoml property could be two machines, one on which the
inultiplication instruction is efficient while the acklition instruction is ine nat,
axl another with these efficiencies reversed. To decide which of the following
algorithms to we inorder to double an integer depemdls on the machine we will
use. Ignorlag constant scaling these algorithms become equivalent, aul we may
think of any instruction as taking “one unit of time” for simplicity of analysis.

   

 

Algorithm 1 DoubleByFactor{m)
1 return 2-n

 

 

Algorithm 2 DoubleByAddition(n)
1 return n+n

 

 

Example 2.1, (Bubble Sort)
 

Algorithm 3 ButhieSort( A, ....n —

 

 

1 for i from () to n —2 do
a suapped = Falke

&
&

  

if not swapped then
& Break

& return A

 

Space Complexity:

The input takes a space of 1.
How much additional space did we ue? we only used i, 9 amd a flag. This
is 4 constant amount, lets say C. A constant is Of1), thus the total space
complexity is $(n) =n + O(1) = O(n).

 

Beat Time Complexity:

 

What is the “best case ecenano”? If the input A is already sorted when the
program starts. In this ease, the Por loop in row $ will nun 1 times, each time
with a constant cost for each execution, s the running time will be an +6 for
some constants a, This is a linear function, this Ty ger(n) = O(n).

 

 

 

Worat Time Complexity

The worst case: the array is sorted in reversed order. In that case, the Fer loop
invow 1 will ran times. Within each run, the inner For loop will run i times,
each time with a constant cost for each execution. So the total running time
will be:

+d= (XS) +d=0(n9}

n=+te(n-D+...

     

Twoasrin)

 

 

When we say “time complexity” we mean “worst tinve complexity”. Nor
unally want to guarantee an upper bound on an algorithm’ nnniag the. This
is done by upper bounding (giving a big} boul) ite worst time complexity.

Now we know how to analyse the resources the algorithm takes. But we
would like to analyse only eorreet algorithme. By correctness of an algorithm
we mean it achieves the task at hand. In this case, that the output army is a
sorted version of the input ene.

We will prove this wing an iuvatiant - a property that is kept true in regard
to sone quantity. Here the invariant will be that the suffix (last part) of the
array is sorted, aud the quantity will be this cuflix’s length.

 
3. (If there is time) Explanation on Loop Invariant:
We can prove the correctness of iterative algorithms using an induction, which is proved
on the iterations of the algorithm (each iteration is accounted as the “i” of the induction).
This will be a key topic in your next course in Algorithms, but itis also important for our
course. We can prove an algorithm's correctness using an induction, which is proved on
the iterations of the algorithm (each iteration is accounted as the “? of the induction).
The computer science way to call this sortof proofis a “loop invariant”, which we will
now cover more precisely.

Aloop invariantis a property of a program loop thatis true before (and after)each
iteration. Itis a logical assertion, sometimes checked within the code by an assertion
call. Knowing a code's invariants is essential in understanding the effect of a loop and
proving its correctness. There are three parts of a loop invariant that we need to show:

1. Initialization: It is true prior to the first iteration of the loop.
2. Maintenance: If it is true before an iteration of the loop, it remains true before the

next iteration.
3. Termination: When the loop terminates, the invariant gives usa Useful property
that helps show that the algorithm is correct.

We'll see some examples:

1. Example 1:
int 3 = 9;

for(int i=0; i<l0; i++)

3-7

. ‘wea ria nt?

 

What is the bop invariant?
max is always maximum among the first ielments of array A.
\input{tex/texlib/tail}
