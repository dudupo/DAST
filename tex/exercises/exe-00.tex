\title{Data Structures - Assignment 0} 
\author{Induction and Asymptotic Notations.}
\date{September 2021}
\documentclass{article}


\usepackage[utf8]{inputenc}
\usepackage[a4paper, total={6in, 9in}]{geometry}
\usepackage{braket}
\usepackage{xcolor}
\usepackage{amsmath}
\usepackage{amsfonts}
\usepackage{tikz}
\usepackage{svg}
\usepackage{graphicx}
\usepackage{media9}
\usepackage{float}
\usetikzlibrary{calc}
\usepackage{array}
\usepackage[ruled,vlined,linesnumbered]{algorithm2e}

\usepackage[
backend=biber,
style=alphabetic,
sorting=ynt
]{biblatex}



\newcommand{\commentt}[1]{\textcolor{blue}{ \textbf{[COMMENT]} #1}}
\newcommand{\ctt}[1]{\commentt{#1}}
\newcommand{\prb}[1]{ \mathbf{Pr} \left[ {#1} \right]}
\newcommand{\onotation}[1]{\(\mathcal{O} \left( {#1}  \right) \)}
\newcommand{\ona}[1]{\onotation{#1}}
\newcommand{\nvar}[2]{ \( #1_{1}, #1_{2} ... #1_{#2}  \) }
%\newenvironment{proof}[0]{\paragraph{Proof.}}{}
%\newenvironment{remark}[0]{\textit{remark}}{}
\newenvironment{cor}[0]{\paragraph{Corollary.}}{}
\newenvironment{example}[0]{\paragraph{Example.}}{}
%\newenvironment{thm}[0]{\paragraph{Theorem.}}{}
\newtheorem{prop}{Proposition}
\newtheorem{ex}{Exercise}
\newtheorem{sol}{Solution}
\newtheorem{theorem}{Theorem} 		
\newtheorem{thm}{Theorem}[section]
\newtheorem{conj}[thm]{Conjecture} 	
\newtheorem{lemma}[thm]{Lemma}
\newtheorem{corollary}[thm]{Corollary} 
\newtheorem{claim}[thm]{Claim}
\newtheorem{proposition}[thm]{Proposition}
\newtheorem{definition}{Definition} 
\newtheorem{remark}{Remark}
 


% \addbibresource{sample.bib} %Import the bibliography file
\begin{document}    
\maketitle


\paragraph{Exercise 1. Induction.}Prove the following statements by induction:
\begin{enumerate}
    \item \( 2^n \le n!\) for \(n\ge 2\).
    \item \(\prod^{n}_{j=1}{\left(1 +\frac{1}{j^2}\right)} \le n+1\).
    \item Prove that \( \displaystyle 1^3+2^3+3^3+\cdots+ n^3=\frac{n^2(n+1)^2}4 \).
    \item For a convex polygon with n sides, the number of diagonals is \(\frac{1}{2}n\left(n-3\right)\).
    \item For a binary tree (not necessarily complete) of height \(h\), the number of nodes is less then \(2^{h+1} -1 \).
    \item Prove that for every positive integer \( n \) there exist positive integers \( a_{11} \),  \( a_{21} \), \( a_{22} \), \( a_{31} \), \( a_{32} \), \( a_{33} \), \( \dots \), \( a_{n1} \), \( a_{n2} \), \( \dots \), \( a_{nn} \) such that 
\[ a_{11}^2=a_{21}^2+a_{22}^2=a_{31}^2+a_{32}^2+a_{33}^2=\cdots=a_{n1}^2+\cdots+a_{nn}^2.\]
\end{enumerate}


\paragraph{Exercise 2. O-notation.}
\begin{enumerate}
\item Prove that \(2^{\sqrt{n}} = \Omega\left(n^k\right)\) for every \(k \in \mathbb{R}^{+}\). 
\item Prove that for every pair of functions \(f,g : \mathbb{N} \rightarrow \mathbb{R}^{+}\) \begin{equation*}
     O \left( f + g \right) = O \left( \max{\{f,g\}} \right)
\end{equation*} 


\item Let \(\zeta_r\) be the area of \(r\)-radius circle. Define a \textit{tiling} as set of non overlapping rectangles with unit-length side such that each rectangle is inside the circle (and doesn't cross the border). Define \(\eta_r\) to be the maximum area of \(r\)-radius circle tiling, Prove that \( \zeta_r = \Theta\left(\eta_r\right)\).  
\end{enumerate}

% \paragraph{Exercise 2. Hierarchy Collapsing.}Let assume the following false statement,  suppose there exists an index \(k \in \mathbb{N}\) such that \(n^k = \Theta\left(n^{k+1}\right) \). 
% \begin{enumerate}
%     \item show that growth of arbitrary polynomial of degree \(d\) is bounded by constant  \( n^d = \Theta \left(1 \right)  \)
%     \item show that the above holds also for exponential growth, \( 2^n = \Theta \left(1 \right)  \)
% \end{enumerate}


% \paragraph{Exercise 3. Arithmetic Mean Behavior.}Prove or disprove,  Let \(f:\mathbb{N} \rightarrow \mathbb{R}^{+}\) be a function such that: \( f(n) = \Theta\left(\frac{1}{n}\sum^{n}_{i=1}{f(i)}\right)\), let \(M\) be a set of \(log\left(n\right)\) arbitrary points, in the range \( \{1,2...,n\}\).

% Does the following bound hold? \begin{equation*} 
%     f(n) = \Theta\left(\frac{1}{n}\sum^{n}_{i=1, i\notin M }{f(i)}\right)
% \end{equation*} 


% \printbibliography 
\end{document}