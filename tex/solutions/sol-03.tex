
\title{Data Structures - Solution 3} 
\author{Induction and Asymptotic Notations.}
\date{September 2021}
\documentclass{article}


\usepackage[utf8]{inputenc}
\usepackage[a4paper, total={6in, 9in}]{geometry}
\usepackage{braket}
\usepackage{xcolor}
\usepackage{amsmath}
\usepackage{amsfonts}
\usepackage{tikz}
\usepackage{svg}
\usepackage{graphicx}
\usepackage{media9}
\usepackage{float}
\usetikzlibrary{calc}
\usepackage{array}
\usepackage[ruled,vlined,linesnumbered]{algorithm2e}

\usepackage[
backend=biber,
style=alphabetic,
sorting=ynt
]{biblatex}



\newcommand{\commentt}[1]{\textcolor{blue}{ \textbf{[COMMENT]} #1}}
\newcommand{\ctt}[1]{\commentt{#1}}
\newcommand{\prb}[1]{ \mathbf{Pr} \left[ {#1} \right]}
\newcommand{\onotation}[1]{\(\mathcal{O} \left( {#1}  \right) \)}
\newcommand{\ona}[1]{\onotation{#1}}
\newcommand{\nvar}[2]{ \( #1_{1}, #1_{2} ... #1_{#2}  \) }
%\newenvironment{proof}[0]{\paragraph{Proof.}}{}
%\newenvironment{remark}[0]{\textit{remark}}{}
\newenvironment{cor}[0]{\paragraph{Corollary.}}{}
\newenvironment{example}[0]{\paragraph{Example.}}{}
%\newenvironment{thm}[0]{\paragraph{Theorem.}}{}
\newtheorem{prop}{Proposition}
\newtheorem{ex}{Exercise}
\newtheorem{sol}{Solution}
\newtheorem{theorem}{Theorem} 		
\newtheorem{thm}{Theorem}[section]
\newtheorem{conj}[thm]{Conjecture} 	
\newtheorem{lemma}[thm]{Lemma}
\newtheorem{corollary}[thm]{Corollary} 
\newtheorem{claim}[thm]{Claim}
\newtheorem{proposition}[thm]{Proposition}
\newtheorem{definition}{Definition} 
\newtheorem{remark}{Remark}
 


% \addbibresource{sample.bib} %Import the bibliography file
\begin{document}    
\maketitle



\paragraph{Exercise 1. Cow Photography.}
The cows are in a particularly mischievous mood today!  All Farmer John
wants to do is take a photograph of the cows standing in a line, but they
keep moving right before he has a chance to snap the picture.

Specifically, each of FJ's \(N\)  cows has a unique
integer ID number.  FJ wants to take a picture of the cows standing in
a line in a very specific ordering, represented by the contents of an
array \( A[1...N]\), where \( A[j]\) gives the ID number of the \(j\)th cow in the
ordering.  He arranges the cows in this order, but just before he can
press the button on his camera to snap the picture, a group of zero or
more cows (not necessarily a contiguous group) moves to a set of new
positions in the lineup.  More precisely, a group of zero or more cows
steps away from the line, with the remaining cows shifting over to
close the resulting gaps in the lineup.  The cows who stepped away
then re-insert themselves at different positions in the lineup (not
necessarily at the locations they originally occupied).  Frustrated
but not deterred, FJ again arranges his cows according to the ordering
in \(A\), but again, right before he can snap a picture, a different group
of zero or more cows moves to a set of new positions in the lineup.

The process above repeats for a total of five photographs before FJ gives
up.  Given the contents of each photograph, see if you can reconstruct the
original intended ordering \(A\).  Each photograph shows an ordering of the
cows that differs from \(A\) in that some group of zero or more cows has moved.
However, a cow only moves in at most one photograph: if a cow is part of
the group that moves in one photograph, she will not actively move in any
of the other four photographs (although she could end up at a different
index as a consequence of other cows around her moving, of course).

\paragraph{Solution.} Consider two cows labeled A and B. Suppose that in the original ordering, A came before B. Notice that in the five photographs, neither A nor B moved in at least three of them (that is, A and B must still be in the correct relative order in at least three of the photographs). Given these five photographs, we can just compare the number of times A came before B; if it is three or more, then A is before B in the original ordering, otherwise B is before A in the original ordering. This provides us with a fast comparison function between any two cows. Now all that is left is sorting all the cows using this comparator.

\paragraph{Exercise 2. O-notation.}
\begin{enumerate}
\item Prove that \(2^{\sqrt{n}} = \Omega\left(n^k\right)\) for every \(k \in \mathbb{R}^{+}\). 
\item Prove that for every pair of functions \(f,g : \mathbb{N} \rightarrow \mathbb{R}^{+}\) \begin{equation*}
     O \left( f + g \right) = O \left( \max{\{f,g\}} \right)
\end{equation*} 


\item Let \(\zeta_r\) be the area of \(r\)-radius circle. Define a \textit{tiling} as set of non overlapping rectangles with unit-length side such that each rectangle is inside the circle (and doesn't cross the border). Define \(\eta_r\) to be the maximum area of \(r\)-radius circle tiling, Prove that \( \zeta_r = \Theta\left(\eta_r\right)\).  
\end{enumerate}

% \paragraph{Exercise 2. Hierarchy Collapsing.}Let assume the following false statement,  suppose there exists an index \(k \in \mathbb{N}\) such that \(n^k = \Theta\left(n^{k+1}\right) \). 
% \begin{enumerate}
%     \item show that growth of arbitrary polynomial of degree \(d\) is bounded by constant  \( n^d = \Theta \left(1 \right)  \)
%     \item show that the above holds also for exponential growth, \( 2^n = \Theta \left(1 \right)  \)
% \end{enumerate}


% \paragraph{Exercise 3. Arithmetic Mean Behavior.}Prove or disprove,  Let \(f:\mathbb{N} \rightarrow \mathbb{R}^{+}\) be a function such that: \( f(n) = \Theta\left(\frac{1}{n}\sum^{n}_{i=1}{f(i)}\right)\), let \(M\) be a set of \(log\left(n\right)\) arbitrary points, in the range \( \{1,2...,n\}\).

% Does the following bound hold? \begin{equation*} 
%     f(n) = \Theta\left(\frac{1}{n}\sum^{n}_{i=1, i\notin M }{f(i)}\right)
% \end{equation*} 

\input{tex/texlib/tail}
