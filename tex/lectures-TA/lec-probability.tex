
\documentclass{article}


\usepackage[utf8]{inputenc}
\usepackage[a4paper, total={6in, 9in}]{geometry}
\usepackage{braket}
\usepackage{xcolor}
\usepackage{amsmath}
\usepackage{amsfonts}
\usepackage{tikz}
\usepackage{svg}
\usepackage{graphicx}
\usepackage{media9}
\usepackage{float}
\usetikzlibrary{calc}
\usepackage{array}
\usepackage[ruled,vlined,linesnumbered]{algorithm2e}

\usepackage[
backend=biber,
style=alphabetic,
sorting=ynt
]{biblatex}



\newcommand{\commentt}[1]{\textcolor{blue}{ \textbf{[COMMENT]} #1}}
\newcommand{\ctt}[1]{\commentt{#1}}
\newcommand{\prb}[1]{ \mathbf{Pr} \left[ {#1} \right]}
\newcommand{\onotation}[1]{\(\mathcal{O} \left( {#1}  \right) \)}
\newcommand{\ona}[1]{\onotation{#1}}
\newcommand{\nvar}[2]{ \( #1_{1}, #1_{2} ... #1_{#2}  \) }
%\newenvironment{proof}[0]{\paragraph{Proof.}}{}
%\newenvironment{remark}[0]{\textit{remark}}{}
\newenvironment{cor}[0]{\paragraph{Corollary.}}{}
\newenvironment{example}[0]{\paragraph{Example.}}{}
%\newenvironment{thm}[0]{\paragraph{Theorem.}}{}
\newtheorem{prop}{Proposition}
\newtheorem{ex}{Exercise}
\newtheorem{sol}{Solution}
\newtheorem{theorem}{Theorem} 		
\newtheorem{thm}{Theorem}[section]
\newtheorem{conj}[thm]{Conjecture} 	
\newtheorem{lemma}[thm]{Lemma}
\newtheorem{corollary}[thm]{Corollary} 
\newtheorem{claim}[thm]{Claim}
\newtheorem{proposition}[thm]{Proposition}
\newtheorem{definition}{Definition} 
\newtheorem{remark}{Remark}
 


% \addbibresource{sample.bib} %Import the bibliography file
\begin{document}    
\maketitle
\begin{document}
\ifdefined\BOOK
\else
\setcounter{chapter}{6}
\fi
\chapter{Probability.} 

\section{ Probability Spaces. }

\begin{definition}
  A probability space is defined by a tuple $(\Omega,P)$, where:
  \begin{enumerate} 
    \item $\Omega$ is a set, called the sample space. Any element $\omega\in \Omega$ is an atomic event. Conceptually, we think of atomic events as possible outcomes of our experiment. Any subset $A \subset \Omega$ is an event. 
    \item $P$, called the probability function, is a function that assigns a number in $[0,1]$ to any event, denoted as $P : 2^\Omega \rightarrow [0,1]$, and satisfies:
      \begin{enumerate}
        \item For any event $A \subset \Omega$, $P(A) = \sum_{\omega\in A}P(\omega)$. 
        \item Normalization over the atomic events to $1$, which means $\sum_{\omega\in\Omega}P(\omega)~=~1$.
      \end{enumerate}
  \end{enumerate}
\end{definition}
\begin{example}
  Consider a dice rolling, where each of the faces is indexed by $1,2,3,4,5,6$ and has an equal chance of being rolled. Therefore, our atomic events are associated with the rolling result, and $P$ is defined as $P(\omega) = \frac{1}{6}$ for any such atomic event.
  An example of an event can be $A=$ ''the dice falls on an even number''. The probability of this outcome is:
  \begin{equation*}
    \begin{split}
      P(A)= \sum_{\omega\in A}{ P(\omega) } = P(\{2\}) + P(\{4\}) + P(\{6\}) = 3\cdot \frac{1}{6} = \frac{1}{2} 
    \end{split}
  \end{equation*}
\end{example}

%\begin{claim}
%Probability function  satisfies the following properties:
%\begin{enumerate}
%  \item $P(\emptyset) = 0$.
%  \item Monotonic, If $A \subset B \subset \Omega$ then $P(A) \le P(B)$.
%  \item Union Bound, $P(A \cup B) \le P(A) + P(B)$.
%  \item Additivity for disjointness events. If $A\cap B = \emptyset$ then $P(A \cup B) = P(A) + P(B)$.
%  \item Denote by $\bar{A}$ the complementary event of $A$, which means $A\cup\bar{A} = \Omega$. Then, $P(\bar{A}) = 1 - P(A)$.
%\end{enumerate}
%\end{claim}
%
\begin{claim}
The probability function satisfies the following properties:
\begin{enumerate}
  \item $P(\emptyset) = 0$.
  \item Monotonicity: If $A \subset B \subset \Omega$, then $P(A) \le P(B)$.
  \item Union Bound: $P(A \cup B) \le P(A) + P(B)$.
  \item Additivity for disjoint events: If $A\cap B = \emptyset$, then $P(A \cup B) = P(A) + P(B)$.
  \item Complementarity: Denote by $\bar{A}$ the complementary event of $A$, which means $A\cup\bar{A} = \Omega$. Then, $P(\bar{A}) = 1 - P(A)$.
\end{enumerate}
\end{claim}

\begin{example}
  Let's proof the additivity of disjointness property. Let $A,B$ disjointness events, so $A \cap B = \emptyset$ then 
  
  \begin{equation*}
    \begin{split}
      P(A\cup B) &= \sum_{w \in A \cup B}P(w) \\ 
      &= \overbrace{\sum_{w \in A, w \notin{B}}P(w)}^{P(A)} + \overbrace{\sum_{w \in B, w \notin A}P(w)}^{P(B)}  +\overbrace{ \sum_{w \in A, w \in  B}P(w) }^{ 0 } \\ 
      &= P(A) + P(B) 
    \end{split}
  \end{equation*}
\end{example}

\begin{definition}
  Let $(\Omega,P)$ be a probability space. A random variable $X$ on $(\Omega,P)$ is a function $X : \Omega \rightarrow \mathbb{R}$. An indicator, is a random variable defined by an event $A \subset \Omega$ as follows   
  \begin{equation*}
    X(\omega) = \begin{cases}
      1 & \omega \in A \\
      0 & \omega \notin A
    \end{cases}
  \end{equation*}
Sometimes, we will use the notation $\{ X = x \}$ to denote the event $A$ such: 
\begin{equation*}
  \begin{split}
    A = \{ \omega : X(\omega) = x \} := \{ X = x \} 
  \end{split}
\end{equation*}
\end{definition}
\begin{example} \label{example:twodic} Consider rolling a pair of dice. Denote by $X : [6]\times [6] \rightarrow [6] $ the random variable that is set to be the result of the first roll. Let $Y$ be defined in almost the same way, but setting the result of the second die. Namely, if we denote by $\{(i,j)\}$ the atomic event associated with sample $i$ on the first die and $j$ on the second die, then:
  \begin{equation*}
    \begin{split}
      X(\{i,j\}) &= i \\ 
      Y(\{i,j\}) &= j  
    \end{split}
  \end{equation*}
  In addition, one can define the random variable $z$ as the sum, $Z = X+Y$. Since the sum is also a function from $\Omega$ to $\mathbb{R}$, $Z$ is also a random variable. An example of an indicator could be $W$, which gets $1$ if $Z \in \{2, 7, 8\}$.
\end{example}

\begin{definition}
  We will say that two random variable $X,Y : \Omega \rightarrow \mathbb{R}$  are independent if for any $x\in \image X$ and $y \in \image Y$:   
  \begin{equation*}
    \begin{split}
      P(X = x \cap Y = y) &= P(X = x) \cdot P (Y = y) %\Leftrightarrow \\ 
    \end{split}
  \end{equation*}
  \end{definition}
  \begin{example}
    $X,Y$ defined in \Cref{example:twodic} are independent. 
    \begin{equation*}
      \begin{split}
        P(\{X = i\} \cap \{Y = j\} ) &= \sum_{i^{\prime} = i \text{ and } j^{\prime}=j }P(\{(i^{\prime}, j^{\prime})\}) = P(\{ (i,j) \} ) \\ 
        &= \frac{1}{36} = \frac{1}{6} \cdot \frac{1}{6}  = P(X = i)P(Y = j)
      \end{split}
    \end{equation*}
  \end{example}

  \section{ Throwing Keys to Cells. } Imagines that following experiment, we have $m$ cells and $n$ keys (balls, numbers, your favorite object type). We throw each of the keys independently to cells. We would like to analyze how capacity of the cells is distributed.   
 
  \begin{example} \\ 
  %\paragraph{Question.} 
    \begin{enumerate}
      \item What is the probability that the first and the second keys will be thrown to the first cell? 
      \item What is the probability that the first and the second keys will be thrown to the same cell? 
      \item What is the probability that in each cell there is at most one key? 
    \end{enumerate}
    Let us define the indicator $X_{i}^{j}$ which indicate that the $j$th key fallen into the $i$th cell. 
    \begin{enumerate}
      \item So first we been asked whether $X_{1}^{1}\cdot  X_{1}^{2} = 1$
      \item Now we been asked whether there exists $i$ such that $X_{j}^{1}\cdot  X_{j}^{2} = 1$
      \item Now we been asked whether there exists $i$ such that $X_{j}^{1}\cdot  X_{j}^{2} = 1$
    \end{enumerate}
  \end{example}

\begin{definition}
  Let $X : \Omega\rightarrow \mathbb{R}$ be a random variable, the expectation of $X$ is 
  \begin{equation*}
    \begin{split}
      \expp{X} = \sum_{\omega\in \Omega}{X(\omega)P(\omega)} = \sum_{x \in \image X}{ x P(X = x)} 
    \end{split}
  \end{equation*} Observes that if $P$ is distributed uniformly, then the expectation of $X$ is just the arithmetic mean: \begin{equation*}
    \begin{split}
      \expp{X} = \sum_{\omega\in \Omega}{X(\omega)P(\omega)} =  \frac{1}{|\Omega|}\sum_{\omega\in \Omega}{X(\omega)}  
    \end{split}
  \end{equation*}
\end{definition}

\begin{claim}
   The expectation satisfies the following properties:
   \begin{enumerate}
     \item Monotonic, If $X \le Y$ (for any $\omega \in \Omega$) then $\expp{X} \le \expp{Y}$.   
     \item Linearity, for $a,b \in \mathbb{R}$ it holds that $\expp{aX + by} = a\expp{X} + b \expp{Y}$.   
     \item Independently, if $X,Y$ are independent, then $\expp{X\cdot Y} = \expp{X}\cdot \expp{Y}$.  
     \item For any constant $a \in \mathbb{R}$ we have that $\expp{a} = a$. 
   \end{enumerate}
\end{claim}

\begin{proof} 
  \begin{enumerate}
\item Monotonic, if $X \le Y$ then : 
  \begin{equation*}
    \begin{split}
      \expp{X} = \sum_{\omega\in \Omega}{X(\omega)P(\omega)} \le  \sum_{\omega\in \Omega}{Y(\omega)P(\omega)} = \expp{Y}
    \end{split}
  \end{equation*}
\item Linearity,
  \begin{equation*}
    \begin{split}    
      \expp{a X + b Y} &=  \sum_{\omega\in \Omega}{ \left( aX(\omega) + bY(\omega) \right) P(\omega)} \\ 
      &=a\sum_{\omega\in \Omega}{  X(\omega)  P(\omega) } + b \sum_{\omega\in \Omega}{   Y(\omega)  P(\omega)}
    \end{split}
  \end{equation*}
\item Independently, 
  \begin{equation*}
    \begin{split}
      \expp{X Y} &= \sum_{x,y \in \image X \times \image Y}{ xy P(X = x \cap Y = y)} \\
      &= \sum_{x,y \in \image X \times \image Y}{ xy P(X = x )P(Y = y)} \\
      &=\sum_{x \in \image X}{\sum_{y \in \image Y}{ xy P(X = x )P(Y = y)}} \\
      &=\sum_{x \in \image X}{xP(X=x)\sum_{y \in \image Y}{ y P(Y = y)}} \\
      &=\sum_{x \in \image X}{xP(X=x) \expp{Y} }  \\ 
      &=\expp{X} \expp{Y}% \sum_{x \in \image X}{xP(X=x) }  \\ 
    \end{split}
  \end{equation*}
\item Let $X$ be the random variable which is also the constant function $X(\omega) = a$ for any $\omega \in \Omega$. Then we have that
  \begin{equation*}
    \begin{split}
      \expp{X} & = \sum_{\omega\in \Omega}{X(\omega)P(\omega)} \\ & = \sum_{\omega\in \Omega}{a P(\omega)} = a \cdot 1 = a  
    \end{split}
  \end{equation*}
  \end{enumerate}
\end{proof}

\begin{example} Let $X$ be an indicator of event $A$, what are $\expp{X}$ and $\expp{X^{2}}$? Recall that $X(\omega) = 1$ only if $\omega \in A$ and $0$ otherwise, thus: 
  \begin{equation*}
    X^{k}(\omega) = \begin{cases}
      1^{k} = 1 & \omega \in A \\
      0^{k} = 0 & \text{ else }
    \end{cases} \Rightarrow X^{k}(\omega) = X(\omega)
  \end{equation*}
  Therefore, 
  \begin{equation*}
    \begin{split}
      \expp{X^{k}} = \sum_{\omega \in \Omega}{ X^{k}(\omega)  P(\omega)  } = \sum_{\omega \in \Omega}{ X(\omega)  P(\omega)  }=\expp{X} 
    \end{split}
  \end{equation*}
\end{example}
\begin{example}
  \ctt{How many keys trowed into the same cell as the first key thrown to?}
\end{example}

\begin{algorithm}
    	let B[0 : n – 1] be a new array \\
	\For{ $i \leftarrow [0, n – 1]$}{
	   make $B_{i}$ an empty list
       	}
	\For{ $i \leftarrow [1, n]$}{
	    insert $A_{i}$ into list $B_{ \lfloor n A_{i} \rfloor} ]$
       	}
	\For{ $i \leftarrow [0, n – 1]$}{
	    sort list $B_{i}$
       	}
	concatenate the lists $B_{0}, B_{1}, .. , B_{n – 1}$ together and\\
	return the concatenated lists
\caption{bucket-sort($A$, $n$)}
  \end{algorithm}

  Denote by $X_{i} : [n] \rightarrow [n]$ then random variable that counts the number of elements fallen in the $i$th bucket. The Expectation of the sorting running time is:   
  
  \begin{equation*}
    \begin{split}
    \expp{T} &= \expp{  \text{ Inserting into buckets  }   +   \sum_{i} { \text{Sorting $i$th bucket  } }} \\ 
    & = \expp{ \Theta(n) +   \sum_{i} { X_{i}^{2}  }} = \Theta(n) +  \sum_{i}{ \expp{X_{i}^{2}} }  \\
  \expp{X_{i}^{2}} &= \expp{\left( \sum X_{i}^{j} \right)^{2}} = \expp{\sum_{j,j^{\prime}} X_{i}^{j} X_{i}^{j^{\prime}} } =  \sum_{j, j^{\prime}} \expp{X_{i}^{j} X_{i}^{j^{\prime}}} \\ 
  & = \sum_{j \neq j^{\prime}} \expp{X_{i}^{j} X_{i}^{j^{\prime}}} + \sum_{j} \expp{X_{i}^{j} X_{i}^{j}} \\
    & =  \sum_{j \neq j^{\prime}} \expp{X_{i}^{j} X_{i}^{j^{\prime}}} + \sum_{j} \expp{X_{i}^{j}}  \\
        & = 2 { n \choose 2  } \left( \frac{1}{n} \right)^{2} +  n\cdot \frac{1}{n} \\ 
        & = \frac{n-1}{n} + 1  = 2 - \frac{1}{n} \Rightarrow \expp{T} = \Theta(n) + n\left( 2 - \frac{1}{n} \right) = \Theta(n)
    \end{split}
  \end{equation*}


% \printbibliography 
\end{document}


