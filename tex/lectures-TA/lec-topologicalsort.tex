
\documentclass{article}


\usepackage[utf8]{inputenc}
\usepackage[a4paper, total={6in, 9in}]{geometry}
\usepackage{braket}
\usepackage{xcolor}
\usepackage{amsmath}
\usepackage{amsfonts}
\usepackage{tikz}
\usepackage{svg}
\usepackage{graphicx}
\usepackage{media9}
\usepackage{float}
\usetikzlibrary{calc}
\usepackage{array}
\usepackage[ruled,vlined,linesnumbered]{algorithm2e}

\usepackage[
backend=biber,
style=alphabetic,
sorting=ynt
]{biblatex}



\newcommand{\commentt}[1]{\textcolor{blue}{ \textbf{[COMMENT]} #1}}
\newcommand{\ctt}[1]{\commentt{#1}}
\newcommand{\prb}[1]{ \mathbf{Pr} \left[ {#1} \right]}
\newcommand{\onotation}[1]{\(\mathcal{O} \left( {#1}  \right) \)}
\newcommand{\ona}[1]{\onotation{#1}}
\newcommand{\nvar}[2]{ \( #1_{1}, #1_{2} ... #1_{#2}  \) }
%\newenvironment{proof}[0]{\paragraph{Proof.}}{}
%\newenvironment{remark}[0]{\textit{remark}}{}
\newenvironment{cor}[0]{\paragraph{Corollary.}}{}
\newenvironment{example}[0]{\paragraph{Example.}}{}
%\newenvironment{thm}[0]{\paragraph{Theorem.}}{}
\newtheorem{prop}{Proposition}
\newtheorem{ex}{Exercise}
\newtheorem{sol}{Solution}
\newtheorem{theorem}{Theorem} 		
\newtheorem{thm}{Theorem}[section]
\newtheorem{conj}[thm]{Conjecture} 	
\newtheorem{lemma}[thm]{Lemma}
\newtheorem{corollary}[thm]{Corollary} 
\newtheorem{claim}[thm]{Claim}
\newtheorem{proposition}[thm]{Proposition}
\newtheorem{definition}{Definition} 
\newtheorem{remark}{Remark}
 


% \addbibresource{sample.bib} %Import the bibliography file
\begin{document}    
\maketitle
\begin{document}
\ifdefined\BOOK
\else
\setcounter{chapter}{9}
\fi
\chapter{Strongly Connected Components and Topological Sort.} 


\section{Topological Sort}


\ifdefined\DEFconnectivity
\else
\providecommand{\DEFconnectivity}{DEFconnectivity}


\begin{definition} (connectivity)
  \label{def:connectivity}
\begin{enumerate}
  \item Let $ G =\left( V,E \right)$ be a non-directed graph. A \textbf{connected component} of $G$ is a subset $U \subseteq V$ of maximal size in which there exists a path between every two vertices. 
\item A non-directed graph $G$ is said to be a \textbf{connected} graph if it only has one connected component.
\item Let $ G = \left(V, E\right)$ be a directed graph. A \textbf{strongly connected component} of $G$ is a subset $U \subseteq V$ of maximal size in which for any pair of vertices $u,v \in U$ there exist both directed path from $u$ to $v$ and a directed path form $v$ to $u$.   
\end{enumerate}
\end{definition}
\fi




\subsection{Depth First Search (DFS)}
As its name implies, depth-first search searches "deeper" in the graph whenever possible. Depth-first search explores edges out of the most recently discovered vertex $v$ that still has unexplored edges leaving it. Once all of $ v$'s edges have been explored, the search "backtracks" to explore edges leaving the vertex from which $v$ was discovered. This process continues until all vertices that are reachable from the original source vertex have been discovered. If any undiscovered vertices remain, then depth-first search selects one of them as a new source, repeating the search from that source. The algorithm repeats this entire process until it has discovered every vertex.


\begin{minipage}[t]{0.46\textwidth}

  \begin{algorithm}[H]
   \textbf{DFS( $G$):}\\
    \For {$v\in V$}{
 	$v$.visited $\leftarrow False$
    }
    time $ \leftarrow 1 $\\
    \For {$v\in V$}{
      \If { not $v$.visited } {
	$\pi \left( v \right)  \leftarrow $ null \\ 
	Explore( $G,v$ ) 
     } 
   }

  \end{algorithm}

  \begin{algorithm}[H]
    \textbf{Explore($G,v$):}\\
     Previsit($v$)\\
    \For {$\left( v,u \right) \in E  $}{
      \If { not $u$.visited } {
	$ \pi \left( u \right) \leftarrow v $ \\ 
	Explore( $G, u$ ) 
      }
    }
    Postvisit($v$)
  \end{algorithm}
  \end{minipage}
   \hfill
   \begin{minipage}[t]{0.46\textwidth}
  \begin{algorithm}[H]
    \textbf{Previsit($v$):}\\
    pre($v$) $\leftarrow $ time \\
    time $\leftarrow$ time $+1$
  \end{algorithm}
  \begin{algorithm}[H]
   \textbf{Postvisit($v$):} \\
    post($v$) $\leftarrow $ time \\
    time $\leftarrow$ time $+1$
  \end{algorithm}

  \end{minipage}

\paragraph{Properties of depth-first search.} Depth-first search yields valuable information about the structure of a graph. Perhaps the most basic property of depth-first search is that the predecessor subgraph $G_{\pi}$ does indeed form a forest of trees since the structure of the depth-first trees exactly mirrors the structure of recursive calls of explore-function. That is, $u$ = $\pi\left( v \right)$ if and only if explore($G, v$) was called during a search of $ u$'s adjacency list.


Additionally, vertex $v$ is a descendant of vertex $u$ in the depth-first forest if and only if $v$ is discovered during the time in which $u$ is gray. Another important property of depth-first search is that discovery and finish times have a parenthesis structure. If the explore procedure were to print a left parenthesis "$(u$" when it discovers vertex $u$ and to print a right parenthesis "$u)$" when it finishes $u$, then the printed expression would be well-formed in the sense that the parentheses are properly nested.

The following theorem provides another way to characterize the parenthesis structure.

\begin{theorem}[Parenthesis theorem]
In any depth-first search of a (directed or undirected) graph $G = (V, E)$, for any two vertices $u$ and $v$, exactly one of the following three conditions holds:

\begin{enumerate}
  \item the intervals [pre($u$), post($u$)] and [pre($v$), post($v$)] are entirely disjoint, and neither $u$ nor $v$ is a descendant of the other in the depth-first forest.
  \item the interval [pre($u$), post($u$)] is contained entirely within the interval [pre($v$), post($v$)], and $u$ is a descendant of $v$ in a depth-first tree, or
  \item the interval [pre($v$), post($v$)] is contained entirely within the interval [pre($u$), post($u$)], and $v$ is a descendant of $u$ in a depth-first tree.
  \end{enumerate}
\end{theorem}
  \begin{proof} 
      Assume without loss of generality that pre($u$) $<$ pre($v$). Split to the following:
	\begin{enumerate}
	  \item Either pre($v$) $<$ post($u$). In that case, we will prove, by induction on $\text{pre}(v)-\text{pre}(u)$, that for any $u,v$ satisfies the relations, the third case holds. 
	    \begin{enumerate}
	      \item Base. $\text{pre}(v)-\text{pre}(u) = 1$, Then clearly $\{u,v\}$, i.e $v$ is a direct child of $u$. Showing that the value of $\text{post}(v)$ has to be set before  $\text{post}(v)$ is left as an exercise. 
	      \item Assumption. Assume correctness for any $\text{pre}(v)-\text{pre}(u) < t < \text{post}(u)$.
	      \item Step. Consider $t > 1$ such  $\text{pre}(v)-\text{pre}(u) = t$. Since $t > 1$ there is must to be vertex $w$ for which $\pre(u) < \text{pre}(w) < \text{pre(v)} = t$. Splits again:  
		\begin{enumerate}
		  \item Either $\text{post}(w) > \text{pre}(v)$. Observes that: \begin{equation*}
		      \begin{split}
\text{pre}(v) - \text{pre}(w) < \text{pre}(v) - \text{pre}(u) = t
		      \end{split}
		    \end{equation*}
		     and also: \begin{equation*}
		       \begin{split}
\text{pre}(w) - \text{pre}(u) < \text{pre}(v) - \text{pre}(u) = t
		       \end{split}
		     \end{equation*}
		      Therefore by the induction assumption: \begin{equation*}
			\begin{split}
[\pre(v),\post(v)] \subset [\pre(w),\post(w)]\subset [\pre(u),\post(u)]
			\end{split}
		      \end{equation*} and in addition $w$ is a descendant of $u$ and $v$ is a descendant of $w$. Hence $v$ is a descendant of $u$ and the third case holds. 
		    \item Or $\text{post}(w) < \text{pre}(v)$ for any $w$ satisfies $\pre(u) < \pre(w) < \pre(v)$. That means that any call for $\textbf{Explore}(G,w)$ at line $6$ (over suited $w$'s) returned, and at time $t$ a new child of $u$ has been discovered (otherwise it is contradiction for $\post(u) > t$), namely $v$ is a direct child of $u$ and we back to the base.
		\end{enumerate}
	    \end{enumerate}
	    so that $v$ was discovered while $u$ was still gray, which implies that $v$ is a descendant of $u$. Moreover, since $v$ was discovered after $u$, all of its outgoing edges are explored, and $v$ is finished before the search returns to and finishes $u$. In this case, therefore, the interval [pre($v$), post($v$)] is entirely contained within the interval [pre($u$), post($u$)].
	  \item Or, post($u$) $<$ pre($v$), and by definition, pre($u$) $<$ post($u$) $<$ pre(v) $<$ post($v$), and thus the intervals [pre($u$), post($u$)] and [pre($v$), post($v$)] are disjoint. Because the intervals are disjoint, neither vertex was discovered while the other was gray, and so neither vertex is a descendant of the other.
	\end{enumerate}
\end{proof}

\begin{corollary}
  Vertex $v$ is a proper descendant of vertex $u$ in the depth-first forest for a (directed or undirected) graph $G$ if and only if pre($u$) $<$ pre($v$) $<$ post($v$) $<$ post($u$).
\end{corollary}
  
  %%\paragraph{Problem. Two Chess Pieces.}
 %Moti has a tree with n nodes. He is willing to share it with you, which means you can operate on it.

%Initially, there are two chess pieces on the node 1 of the tree. In one step, you can choose any piece, and move it to the neighboring node. You are also given an integer $d$. You need to ensure that the distance between the two pieces doesn't ever exceed $d$.

%Each of these two pieces has a sequence of nodes which they need to pass in any order, and eventually, they have to return to the root. As a curious boy, he wants to know the minimum steps you need to take.

%\paragraph{Solution.} We can find that for any $d$-th ancestor of some $b_i$, the first piece must pass it some time. Otherwise, we will violate the distance limit. The second piece must pass the $d$-th ancestor of each $b_i$ as well. Then we can add the $d$-th ancestor of each $a_i$ to the array $b$, and add the $d$-th ancestor of each $b_i$ to the array $a$.
%
%Then we can find now we can find a solution that each piece only needs to visit its nodes using the shortest route, without considering the limit of $d$, and the total length can be easily computed. We can find that if we adopt the strategy that we visit these nodes according to their DFS order(we merge the array of $a$ and $b$, and sort them according to the DFS order, if the first one is from $a$, we try to move the first piece to this position, otherwise use the second piece), and move the other piece one step closer to the present piece only if the next step of the present piece will violate the distance limit, then we can ensure the movement exactly just let each piece visit its necessary node without extra operations.
%



% \printbibliography 
\end{document}


