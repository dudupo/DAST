\documentclass{article}


\usepackage[utf8]{inputenc}
\usepackage[a4paper, total={6in, 9in}]{geometry}
\usepackage{braket}
\usepackage{xcolor}
\usepackage{amsmath}
\usepackage{amsfonts}
\usepackage{tikz}
\usepackage{svg}
\usepackage{graphicx}
\usepackage{media9}
\usepackage{float}
\usetikzlibrary{calc}
\usepackage{array}
\usepackage[ruled,vlined,linesnumbered]{algorithm2e}

\usepackage[
backend=biber,
style=alphabetic,
sorting=ynt
]{biblatex}



\newcommand{\commentt}[1]{\textcolor{blue}{ \textbf{[COMMENT]} #1}}
\newcommand{\ctt}[1]{\commentt{#1}}
\newcommand{\prb}[1]{ \mathbf{Pr} \left[ {#1} \right]}
\newcommand{\onotation}[1]{\(\mathcal{O} \left( {#1}  \right) \)}
\newcommand{\ona}[1]{\onotation{#1}}
\newcommand{\nvar}[2]{ \( #1_{1}, #1_{2} ... #1_{#2}  \) }
%\newenvironment{proof}[0]{\paragraph{Proof.}}{}
%\newenvironment{remark}[0]{\textit{remark}}{}
\newenvironment{cor}[0]{\paragraph{Corollary.}}{}
\newenvironment{example}[0]{\paragraph{Example.}}{}
%\newenvironment{thm}[0]{\paragraph{Theorem.}}{}
\newtheorem{prop}{Proposition}
\newtheorem{ex}{Exercise}
\newtheorem{sol}{Solution}
\newtheorem{theorem}{Theorem} 		
\newtheorem{thm}{Theorem}[section]
\newtheorem{conj}[thm]{Conjecture} 	
\newtheorem{lemma}[thm]{Lemma}
\newtheorem{corollary}[thm]{Corollary} 
\newtheorem{claim}[thm]{Claim}
\newtheorem{proposition}[thm]{Proposition}
\newtheorem{definition}{Definition} 
\newtheorem{remark}{Remark}
 


% \addbibresource{sample.bib} %Import the bibliography file
\begin{document}    
\maketitle
\newcommand*{\RECITATION}{}%
\newcommand*{\BOOK}{}%

\begin{document}
\chapter{Introduction to Algorithms.}
%, Correctness and Efficiency
\ifdefined\RECITATION
\else
Computer science differs from other scientific disciplines in that it focuses not on solving or making discoveries, but on questioning how good our current understanding is. The fact that one has successfully come up with an idea for a certain problem immediately raises the question of optimality. At the most basic level, we would like to answer what the 'best' program is that exists for a particular problem. To do so, we must have a notation that allows us to determine if an algorithm is indeed solving the task, quantify its performance, and compare it to other algorithms. In this chapter, we introduce this basic notation. The chapter is divided into two main parts: the first is about induction, a mathematical technique for proving claims, and the second presents asymptotic notation, which we use to describe the behavior of algorithms over large inputs.
\fi
\section{Peaks-Finding.}
\begin{example}
  Consider an $n$-length array $A$ such that $A_1, A_2, \ldots, A_n \in \mathbb{R}$. We will say that $A_{j}$ is a \textbf{peak} (local maximum) if it is greater than its neighbors. Namely, $A_{i} \ge A_{i\pm1}$ if $i\pm 1 \in [n]$. Whenever $i\pm 1$ is not in the range $[n]$, we will define the inequality $A_{i} \ge A_{i\pm 1}$ to hold trivially. For example, for $n=1$, $A_{1}=A_{n}$ is always a peak.

  Write an algorithm that, given $A$, returns the position of an arbitrary peak.
\end{example}


\begin{example}{Warming up.} \label{example:func} 
  How many peaks do the following arrays contain?
  \begin{enumerate}
    \item $A[i] = 1$   $\forall i \in [n]$
    \item $A[i] = \begin{cases}
        i & i < n/2 \\
        n/2 - i & \text{else}
      \end{cases}$
    \item $A[i] = i $  $\forall i \in [n]$
  \end{enumerate}
\end{example}



\begin{figure}[h]
  \label{fig:peaks}
\begin{tikzpicture}
    %\draw[->]  (0,0)coordinate(O) -- (14,0) node[anchor=north] {$n$};
    %\draw[->]  (0,0) -- (0,5);

    \draw[domain=0:4,smooth,variable=\x,blue,name path=c1] plot ({\x},{ 1 })node[right]{$1$};
    \draw[domain=5:7.05,smooth,variable=\x,blue,name path=c2] plot ({\x},{ \x-5});
    \draw[domain=7:9,smooth,variable=\x,blue,name path=c3] plot ({\x},{ 2-(\x-7)})node[right]{$2$};
    \draw[domain=10:14,smooth,variable=\x,blue,name path=c4] plot ({\x},{\x-10})node[right]{$3$};
    \draw circle(2pt) (14,4);
\fill[red,name intersections={of=c2 and c3}]
    (intersection-1) circle (2pt);
 %   (intersection-2) circle (2pt)
  %      (intersection-3) circle (2pt) ;
\end{tikzpicture}
\caption{The arrays.} 
\end{figure}




\section{Naive solution.}
To better understand the problem, let's first examine a simple solution before proposing a more intriguing one. Consider the algorithm examining each of the items $A_i$ one by one.
\begin{algorithm}
\KwResult{returns a peak of \(A_1 ... A_n \in \mathbb{R}^n \)  }
\caption{naive peak-find alg.}
 \For{ \(i \in [n] \) } { 
   \If { $A_{i}$ is a peak }{
        \Return $i$
        }
    } 
  \label{alg:simp}
\end{algorithm}
\paragraph{Correctness.}
We will say that an algorithm is correct with respect to a given task if it computes the task for any input. Let's prove that the above algorithm is doing the job.

\begin{proof}
  Assume, trough contradiction, that there exists an $n$-length array $A$ such that the \Cref{alg:simp} fails to find one of its peaks. In particular, the algorithm either returns $j^{\prime} \in [n]$ such that $A_{j^{\prime}}$ is not a peak or does not return at all (never reaches line (3)).
\begin{enumerate}
  \item Let's handle first the case in which returning indeed occurred. Denote by $j$ the first position of a peak in $A$, and note that if the algorithm gets to line (2) in the $j$th iteration then either it returns $j$ or $A_{j}$ is not a peak.
  Hence it must hold that $j^{\prime} < j$. But satisfaction of the condition on line (2) can happen only if $A_{j^{\prime}}$ is a peak, which contradicts the minimality of $j$.
\item In the case that no position has been returned, it follows that the algorithm didn't return in any of the first $j$ iterations and gets to iteration number $j+1$, which means that the condition on line (2) was not satisfied in contradiction to the fact that $A_{j}$ is a peak.
\end{enumerate}
\end{proof}

\paragraph{Running Time.} 
Question, How would you compare the performance of two different algorithms? What will be the running time of the naive peak-find algorithm? At the lecture, you will see a well-defined way to treat such questions, but for the sake of getting the general picture, let's assume that we pay for any comparison quanta of processing time and, overall, checking if an item in a given position is a peak, cost at most $c\in \mathbb{N}$ time, a constant independent on $n$.

Question, In the worst case scenario, how many local checks does peak-finding do? For the third example in~\Cref{example:func}, the naive algorithm will have to check each item, so the running time adds up to at most $c \cdot n$.

\section{Naive alg. recursive version.}
Now, we will show a recursive version of the navie peak-find algorithm for demonstrating how correctness can be proved by induction. 
\begin{algorithm}[h]
\KwResult{returns a peak of \(A_1 ... A_n \in \mathbb{R}^n \)  }
\caption{naive recursive peak-find alg.}
 \If { $A_{1} \ge A[2]$ or $n=1$ }{
   \Return 1
 }
 \Return 1 + peak-find$(A_{2}, .. A_{n})$
 \end{algorithm}

 \begin{claim} \label{claim:subarray} 
Let $A = A_1, \dots, A_n$ be an array, and $A' = A_2, A_3, \dots, A_n$ be the $n-1$ length array obtained by taking all of $A$'s items except the first. If $A_1 \le A_2$, then any peak of $A'$ is also a peak of $A$.
\end{claim}
\begin{proof} 
Let $A^{\prime}_{j}$ be a peak of $A^{\prime}$. Split into cases upon on the value of $j$.

\begin{enumerate}
  \item If $n-1 > j > 1$, then $A^{\prime}_{j} \ge A^{\prime}_{j \pm 1}$, but for any $j \in [2, n-2]$ we have $A^{\prime}_{j} = A_{j+1}$ and therefore $A_{j+1} \ge A_{j+1 \pm 1} \Rightarrow A_{j+1}$ is a peak in $A$.
  \item If $j^{\prime} = 1$, then $A^{\prime}_{1} > A^{\prime}_{2} \Rightarrow A_{2} \ge A_{3}$ and by combining the assumption that $A_{1} \le A_{2}$ we have that $A_{2} \ge A_{1}, A_{3}$. So $A_{2} = A^{\prime}_{1}$ is also a peak.
  \item The last case $j = n-1$ is left as an exercise.
\end{enumerate}
\end{proof}

One can prove a much more general claim by following almost the same argument presented above.

 \begin{claim} \label{claim:subarraymiddle} 
   Let $A = A_1, \dots, A_n$ be an array, and $A' = A_{j+1}, A_{j+2}, \dots, A_n$ be the $n-j$ length array. If $A_j \le A_{j+1}$, then any peak of $A'$ is also a peak of $A$.
\end{claim}

We are ready to prove the correctness of the recursive version by induction using \Cref{claim:subarray}. 

\begin{enumerate}
  \item Base, single element array. Trivial. 
  \item Assumption, Assume that for any $m$-length array, such that $m<n$ the alg returns a peak. 
  \item Step, consider an array $A$ of length $n$. If $A_1$ is a peak, then the algorithm answers affirmatively on the first check, returning $1$ and we are done. If not, namely $A_1 < A_2$, then by using \Cref{claim:subarray} we have that any peak of $A' = A_2, A_3, \dots, A_n$ is also a peak of $A$. The length of $A'$ is $n-1 < n$. Thus, by the induction assumption, the algorithm succeeds in returning on $A'$ a peak which is also a peak of $A$.
\end{enumerate}

\section{An attempt for sophisticated solution.}
We saw that we can find an arbitrary peak at $c\cdot n$ time, which raises the question, can we do better? Do we really have to touch all the elements to find a local maxima? Next, we will see two attempts to catch a peak at logarithmic cost. The first attempt fails to achieve correctness, but analyzing exactly why will guide us on how to come up with both an efficient and correct algorithm.
\begin{algorithm}
\KwResult{returns a peak of \(A_1 ... A_n \in \mathbb{R}^n \)  }
\caption{fail attempt for more sophisticated alg. }
        $ i \leftarrow  \lceil n/2 \rceil $\\
        \If { $A_{i}$ is a peak }{
          \Return $i$
        }
        \Else { 
          \Return $i - 1  + $ find-peak$\left(A_{i},A_{i+1}..A_{n}\right)$
        }
\end{algorithm}

Let's try to 'prove' it.  
\begin{enumerate}
 \item Base, single element array. Trivial. 
  \item Assumption, Assume that for any $m$-length array, such that $m<n$ the alg returns a peak. 
  \item Step. If $A_{n/2}$ is a peak, we're done. What happens if it isn't? Is it still true that any peak of $A_{i},A_{i+1}, \ldots, A_{n}$ is also a peak of $A$? Consider, for example, $A[i] = n - i$.
\end{enumerate}

\section{Sophisticated solution.}
The example above points to the fact that we would like to have a similar claim to \Cref{claim:subarraymiddle} that relates the peaks of the split array to the original one.  
\begin{algorithm}
\KwResult{returns a peak of \(A_1 ... A_n \in \mathbb{R}^n \)  }
\caption{sophisticated alg.}
$ i \leftarrow  \lceil n/2 \rceil $\\
        \If { $A_{i}$ is a peak }{
          \Return $i$
        }
        \ElseIf { $A_{i-1} \le A_{i}$ } { 
          \Return $i + $ find-peak$\left(A_{i+1}..A_{n}\right)$
        }
        \Else { 
          \Return find-peak$\left(A_{1},A_{2},A_{3}..A_{i-1}\right)$
        }
\end{algorithm}
Let's prove correction.
\begin{proof}
  By induction. 
\begin{enumerate}
  \item Base, single element array. Trivial. 
  \item Assumption, Assume that for any $m$-length array, such that $m<n$ the alg returns a peak. 
  \item Step, Consider an array $A$ of length $n$. If $A_{\lceil n/2 \rceil}$ is a peak, then the algorithm answers affirmatively on the first check, returning $\lceil n/2 \rceil$ and we are done. If not, then either $A_{\lceil n/2 \rceil} < A_{\lceil n/2 \rceil - 1}$ or $A_{\lceil n/2 \rceil} < A_{\lceil n/2 \rceil + 1}$. We have already handled the first case, that is, using \Cref{claim:subarraymiddle} we have that any peak of $A' = A_{\lceil n/2 \rceil + 1}, A_{\lceil n/2 \rceil + 2}, \dots, A_n$ is also a peak of $A$. The length of $A'$ is $n/2 < n$. So by the induction assumption, in the case where $A_{\lceil n/2 \rceil} < A_{\lceil n/2 \rceil - 1}$ the algorithm returns a peak. In the other case, we have $A_{\lceil n/2 \rceil} < A_{\lceil n/2 \rceil + 1}$ (otherwise $A_{\lceil n/2 \rceil}$ would be a peak). We leave finishing the proof as an exercise.
\end{enumerate}

\end{proof}


What's the running time? Denote by $T\left( n \right)$ an upper bound on the running time. We claim that $T(n) \le c_{1} \log (n) - c_{2}$. 
\begin{proof}
  By induction. 
\begin{enumerate}
  \item Base. For the base case, $n \le 3$ we get that $c_{1} \log(1) + c_{2} =  c_{2}$ on the other hand only a single check made by the algorithm, so indeed the base case holds (Choosing $c_{2} \ge$ the cost of a single check).
  \item Induction Assumption. Assume that for any $m < n$, the algorithm runs in at most $c_{1} \log(m) + c_{2}$ time.
  \item Step. Notice that in the worst case, $\lceil n/2 \rceil$ is not a peak, and the algorithm calls itself recursively immediately after paying $c_{2}$ in the first check. Hence: \begin{equation*}
      \begin{split}    
	T\left(n\right) & \le c_{2} + T\left(n/2\right) \le c_{2} + c_{1} \log\left( \lceil n/2 \rceil \right) + c_{2}\\
	& = c_{1} - c_{1} + c_{2} + c_{1} \log\left( \lceil n/2 \rceil \right) + c_{2}\\
	& = c_{1} \log (2) +  c \log\left( \lceil n/2 \rceil  \right) + 2c_{2} - c_{1} \\
	& = c_{1} \log\left(2 \left(\lceil n/2 \rceil \right) \right) + 2c_{2} - c_{1}
      \end{split}
    \end{equation*}
    So, choosing $c_{1} > c_{2}$ gives $2c_{2} - c_{1} < c_{2}$ and therefore:
    \begin{equation*}
      \begin{split}
	T\left(n\right) \le c_{1} \log \left( n  \right) + c_{2}  
      \end{split}
    \end{equation*}
\end{enumerate}
\end{proof}




% \section{Time Planning.}
% \begin{enumerate}
%     \item Welcome words, general information, listing the recitation topics \& QA. 7\textbf{m} (section 1).
%     \item Explain (remind) induction and presenting the first first example. 10\textbf{m}.
%     \item Second weak induction example 4\textbf{m}.
%     \item Strong induction 7\textbf{m}.
%     \item Introduce, the big O. explain the object shortly, plot graphs. yet still not formal definitions. 7\textbf{m}.  
%     \item Shoot the definitions of the \(O, \Omega, \Theta\) at one. 7\textbf{m}.
%     \item 3\textbf{m} spare + break. 
%     \item Review again the asymptotic notations, with examples from section 4. Q\&A. 7\textbf{m}
%     \item Lemma 21 7\textbf{m}
%     \item Claim 22 7\textbf{m}
%     \item The claim beneath Claim 22 7\textbf{m}
%     \item Example 23 7\textbf{m} (should take more time, the above claims will pay the tax). 
%     \item Logarithmic Rules 7\textbf{m}. 
% \end{enumerate}
% \section{General Course Information.}
% \begin{enumerate}
%     \item Introduce yourself
%     \item course mail: huji.dast.2022a@gmail.com
%     \item targilim scoring: 0.85 \( \cdot \) Test + 0.15 \( \cdot \) Average(\(N - 2\)). 
% \item  introduction to the course: \begin{enumerate}
%     \item  It’s important \& fun: \begin{enumerate}
%         \item We going to learn some data structures: Heaps, Trees, Hash Tables
%         \item And some algorithms: Sorting, Minimal Spanning Tree, Shortest Path
%     \end{enumerate}
%     \item Doing your homework by yourself is the best way to improve your solving problems skill.
% \end{enumerate}
% \end{enumerate}

% \paragraph{Abstract.} Today we will cover induction, infinite series, and asymptotic notation. These tools will come in handy (in the next couple of weeks) when we want to find the runtime complexity of an algorithm, specifically using the ’Master Theorem’, to give asymptotic bounds for recursion relations, and to prove loop invariants using (finite) induction.


% \printbibliography 
\end{document}
