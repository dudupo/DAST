
\documentclass{article}


\usepackage[utf8]{inputenc}
\usepackage[a4paper, total={6in, 9in}]{geometry}
\usepackage{braket}
\usepackage{xcolor}
\usepackage{amsmath}
\usepackage{amsfonts}
\usepackage{tikz}
\usepackage{svg}
\usepackage{graphicx}
\usepackage{media9}
\usepackage{float}
\usetikzlibrary{calc}
\usepackage{array}
\usepackage[ruled,vlined,linesnumbered]{algorithm2e}

\usepackage[
backend=biber,
style=alphabetic,
sorting=ynt
]{biblatex}



\newcommand{\commentt}[1]{\textcolor{blue}{ \textbf{[COMMENT]} #1}}
\newcommand{\ctt}[1]{\commentt{#1}}
\newcommand{\prb}[1]{ \mathbf{Pr} \left[ {#1} \right]}
\newcommand{\onotation}[1]{\(\mathcal{O} \left( {#1}  \right) \)}
\newcommand{\ona}[1]{\onotation{#1}}
\newcommand{\nvar}[2]{ \( #1_{1}, #1_{2} ... #1_{#2}  \) }
%\newenvironment{proof}[0]{\paragraph{Proof.}}{}
%\newenvironment{remark}[0]{\textit{remark}}{}
\newenvironment{cor}[0]{\paragraph{Corollary.}}{}
\newenvironment{example}[0]{\paragraph{Example.}}{}
%\newenvironment{thm}[0]{\paragraph{Theorem.}}{}
\newtheorem{prop}{Proposition}
\newtheorem{ex}{Exercise}
\newtheorem{sol}{Solution}
\newtheorem{theorem}{Theorem} 		
\newtheorem{thm}{Theorem}[section]
\newtheorem{conj}[thm]{Conjecture} 	
\newtheorem{lemma}[thm]{Lemma}
\newtheorem{corollary}[thm]{Corollary} 
\newtheorem{claim}[thm]{Claim}
\newtheorem{proposition}[thm]{Proposition}
\newtheorem{definition}{Definition} 
\newtheorem{remark}{Remark}
 


% \addbibresource{sample.bib} %Import the bibliography file
\begin{document}    
\maketitle
\begin{document}
\setcounter{chapter}{4}
\chapter{Reserves Recitation.} 



\section{}
Another sorting algorithms, that it's correctness isn't so obivoius.  


\begin{algorithm}
\SetAlgoLined
\KwResult{returns the multiplication \(x\cdot y\) where \(x,y \in \mathbb{F}^{n}_{2}\) }
\For{ $ i \in [n]$} {
  \For{ $ j \in [n]$} {
    \If { $A_{j} < A_{i} $} {
      swap $A_{i} \leftrightarrow A_{j}$
    }
  }
}


\end{algorithm}


%\author{Master theorem and recursive trees.}
% 
%\begin{abstract}
%    One of the standard methods to analyze the running time of algorithms is to express recursively the number of operations that are made. In the following recitation, we will review the techniques to handle such formulation (solve or bound).  
%\end{abstract}
%
\begin{algorithm}
\SetAlgoLined
\KwResult{returns the multiplication \(x\cdot y\) where \(x,y \in \mathbb{F}^{n}_{2}\) }
 \ \\ 
 \If{ \(x,y \in \mathbb{F}_{2}\) }
    { return \(x \cdot y\) } 
 \ \\ 
 
 \Else {
 define \(x_{l} , x_{r} \leftarrow x \) and \(y_{l} , y_{r} \leftarrow x \) \ \ \ \ \ // \( O \left(n\right) \). \\ 
 \ \\ 
 calculate \(z_0 \leftarrow \text{Karatsuba}\left(x_{l},y_{l}\right)\) \\
 \ \ \ \ \ \ \ \ \ \ \ \ \(z_2 \leftarrow \text{Karatsuba}\left(x_{r},y_{r}\right)\) \\ 
 \ \ \ \ \ \ \ \ \ \ \ \ \(z_1 \leftarrow \text{Karatsuba}\left(x_{r} + x_{l} ,y_{l} + y_{r} \right) - z_0 - z_2 \) \\ 
 \ \\
 return \(z_0 + 2^{\frac{n}{2}}z_1 + 2^{n}z_2\) \ \ \ \ \  // \( O \left(n\right) \). 
 }
\end{algorithm}

% \printbibliography 
\end{document}


