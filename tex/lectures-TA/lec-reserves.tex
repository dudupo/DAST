
\documentclass{article}


\usepackage[utf8]{inputenc}
\usepackage[a4paper, total={6in, 9in}]{geometry}
\usepackage{braket}
\usepackage{xcolor}
\usepackage{amsmath}
\usepackage{amsfonts}
\usepackage{tikz}
\usepackage{svg}
\usepackage{graphicx}
\usepackage{media9}
\usepackage{float}
\usetikzlibrary{calc}
\usepackage{array}
\usepackage[ruled,vlined,linesnumbered]{algorithm2e}

\usepackage[
backend=biber,
style=alphabetic,
sorting=ynt
]{biblatex}



\newcommand{\commentt}[1]{\textcolor{blue}{ \textbf{[COMMENT]} #1}}
\newcommand{\ctt}[1]{\commentt{#1}}
\newcommand{\prb}[1]{ \mathbf{Pr} \left[ {#1} \right]}
\newcommand{\onotation}[1]{\(\mathcal{O} \left( {#1}  \right) \)}
\newcommand{\ona}[1]{\onotation{#1}}
\newcommand{\nvar}[2]{ \( #1_{1}, #1_{2} ... #1_{#2}  \) }
%\newenvironment{proof}[0]{\paragraph{Proof.}}{}
%\newenvironment{remark}[0]{\textit{remark}}{}
\newenvironment{cor}[0]{\paragraph{Corollary.}}{}
\newenvironment{example}[0]{\paragraph{Example.}}{}
%\newenvironment{thm}[0]{\paragraph{Theorem.}}{}
\newtheorem{prop}{Proposition}
\newtheorem{ex}{Exercise}
\newtheorem{sol}{Solution}
\newtheorem{theorem}{Theorem} 		
\newtheorem{thm}{Theorem}[section]
\newtheorem{conj}[thm]{Conjecture} 	
\newtheorem{lemma}[thm]{Lemma}
\newtheorem{corollary}[thm]{Corollary} 
\newtheorem{claim}[thm]{Claim}
\newtheorem{proposition}[thm]{Proposition}
\newtheorem{definition}{Definition} 
\newtheorem{remark}{Remark}
 


% \addbibresource{sample.bib} %Import the bibliography file
\begin{document}    
\maketitle
\begin{document}
\setcounter{chapter}{4}
\chapter{Reserves Recitation.} 



\section{More Sorting, More Correctness.}
Until now, all the algorithms that we have seen were, in some sense, intuitive. We could describe in words, step by step, exactly what the algorithm does. For example, bubble and heapsort both bubble up the greatest element among the remaining elements in each iteration. Merge sort divides the task into subtasks on smaller inputs, starting with sorting the first and second halves of the given array, and then merging the sorted subarrays.

We are about to present another $\Theta(n^{2})$-sorting algorithm, whose correctness is not so obvious. The algorithm was developed by Stanley P. Y. Fung, \cite{Simplesort}, who coined its name - "ICan'tBelieveItCanSort" - due to the surprise of having such a simple sorting algorithm. It's worth mentioning that, despite its simplicity, Fung came up with this algorithm in 2021.


\begin{algorithm}
\SetAlgoLined
\KwResult{Sorting $A_{1},A_{2},..A_{n}$ }
\caption{ "ICan'tBelieveItCanSort"  alg.}
\For{ $ i \in [n]$} {
  \For{ $ j \in [n]$} {
    \If { $A_{j} < A_{i} $} {
      swap $A_{i} \leftrightarrow A_{j}$
    }
  }
}


\end{algorithm}

%\begin{claim}
%  \label{claim:first}
%  After the $i$th iteration, $A_{1} \le A_{2} \le A_{3} .. \le A_{i}$ and $A_{i}$ is the maximum of the whole array. 
%\end{claim}
%\begin{proof}
%  By induction on the iteration number $i$. 
%  \begin{enumerate}
%    \item Base. For $i=1$, it is clear that when $j$ reaches the position of the maximal element, an exchange will occur and $A_{1}$ will be set to be the maximal element. Thus, the condition on line (3) will not be satisfied until the end of the inner loop and indeed, we have that $A_{1}$ at the end of the first iteration is the maximum.
%    \item Assumption. Assume the correctness of the claim for any $i^{\prime} < i$. 
%    \item Step. Consider the $i$th iteration. And observes that if $A_{i} = A_{i-1}$ then $A_{i}$ is also the maximal elememnt in $A$, namely no exchange will be made in $i$th iteration, yet $A_{1} \le A_{2} \le .. \le A_{i-1}$ by the induction assumption, thus  $A_{1} \le A_{2} \le .. \le A_{i-1} \le A_{i}$ and $A_{i}$ is the maximal element, so the claim holds in the end of the iteration. 
%      If $A_{i} < A_{i-1}$ then there exists $k \in [1,i-1]$ such $A_{k} > A_{i}$. Set $k$ to be the minimal position for which the inequlity holds. For Convinet denote by $A^{(j)}$ the array in the begging of the $j$th iteration of the inner loop. And let's split to cases according to $j$ value. 
%      \begin{enumerate}
%        \item $j < k$ By definition of $k$, for any $j < k, A^{(1)}_{j} < A^{(1)}_{i}$, Hence in the first $k-1$ iteration no exchange will be made and we can conclude that $A^{(j)}_{l} = A^{(1)}_{l}$ for any $l \in [n]$ and $j \le k$. 
%        \item $j \ge k$ and $j<i+1$, We claim that for each such $j$ an exhange will always occuer. 
%          \begin{claim}
%            For any $j \in [k,i]$ we have that in the end of the $j$th iteration:  
%            \begin{itemize}
%              \item $A^{(j+1)}_{j} = A^{(j)}_{i}$.
%              \item $A^{(j+1)}_{i} = A^{(j)}_{j} = A^{(1)}_{j}$.
%              \item For any $l > j$ and $l \neq i$ we have $A^{(j+1)}_{l} = A^{(1)}_{l}$.
%            \end{itemize}
%          \end{claim}
%          \begin{proof}
%            Observes that the thired section holds trivally by the defination of the algorithm, it doesn't toach any position greater than $j$ in the first $j$ iterations (inner loop) except the $i$th position. So have to prove only the first two bullrts, And again we are going to prove them by induction.  
%            \begin{enumerate}
%              \item Base. $A^{(1)}_{k}$ is greater than $A_{i}$, and by the above case we have that at the begging of the $k$ iteration $A^{(k)}_{k}=A^{(1)}_{k}, A^{(k)}_{i}=A^{(1)}_{k}$. Therefore the condition on line (3) is satisfied, exchange is been made, and $A^{(k+1)}_{k} =A^{(k)}_{i} = A^{(0)}_{i}$ and $A^{(k+1)}_{i} = A^{(k)}_{k}$. Now  So $A^{(k+1)}_{k+1} = A^{(k)}_{k+1} = A^{(0)}_{k+1}$
%              \item Assumption. Assume the correctness of the claim for any $k\ge j^{\prime} < j \le i$. 
%              \item Step. Consider the $j \in (k,i]$ iteration, By the induction assumption we have that $A^{(j)}_{j-1} = A^{(j-1)}_{i}$ and $A^{(j)}_{i} = A^{(j-1)}_{j-1} = A^{(1)}_{j-1}$. On the otherhand, by the induction assumption of \Cref{claim:first}, $j-1 < i \Rightarrow A^{(1)}_{j-1} \le A^{(1)}_{j}$. Combining the thired bullet we obtain that:                
%                \begin{equation*}
%                  \begin{split}
%                    A^{(j)}_{j} = A^{(1)}_{j} \ge A^{(1)}_{j-1} = A^{(j)}_{i}
%                  \end{split}
%                \end{equation*}
%                And therefore, either there is inequalitiy and exhange be made or there is equality, in both cases after the $i$th iteration we have $A^{(j+1)}_{j} = A^{(j)}_{i}$ and $A^{(j+1)}_{i} = A^{(j)}_{j} = A^{(1)}_{j}$.
%            \end{enumerate}
%          \meend{proof}
%        \item $j > i$, So we know that $A^{(i+1)}_{i}$ is the maximal element in $A$, therfore for any $j$ it holds that $A^{(i+1)}_{i}\ge A^{(i)}_{j}$, it follows that no exchange would been made and $A^{(j)}_{i}$ will remain the maximum till the end of the inner loop.    
%      \end{enumerate}
%  \end{enumerate}
%\end{proof}

\begin{claim}
  \label{claim:first}
  After the $i$th iteration, $A_{1} \le A_{2} \le A_{3} .. \le A_{i}$ and $A_{i}$ is the maximum of the whole array. 
\end{claim}
\begin{proof}
  By induction on the iteration number $i$. 
  \begin{enumerate}
    \item Base. For $i=1$, it is clear that when $j$ reaches the position of the maximum element, an exchange will occur and $A_{1}$ will be set to be the maximum element. Thus, the condition on line (3) will not be satisfied for the remaining $j$-iterations of the inner loop. Therefore, at the end of the first iteration, $A_{1}$ is indeed the maximum.
    \item Assumption. Assume the correctness of the claim for any $i^{\prime} < i$. 
    \item Step. Consider the $i$th iteration. And observe that if $A_{i} = A_{i-1}$ then $A_{i}$ is also the maximal element in $A$, namely no exchange will be made in the $i$th iteration, yet $A_{1} \le A_{2} \le .. \le A_{i-1}$ by the induction assumption, thus  $A_{1} \le A_{2} \le .. \le A_{i-1} \le A_{i}$ and $A_{i}$ is the maximal element, so the claim holds in the end of the iteration. 
      If $A_{i} < A_{i-1}$ then there exists $k \in [1,i-1]$ such that $A_{k} > A_{i}$. Set $k$ to be the minimal position for which the inequality holds. For convenience, denote by $A^{(j)}$ the array in the beginning of the $j$th iteration of the inner loop. And let's split into cases according to $j$ value. 
      \begin{enumerate}
        \item $j < k$ By definition of $k$, for any $j < k, A^{(1)}_{j} < A^{(1)}_{i}$, Hence in the first $k-1$ iterations no exchange will be made and we can conclude that $A^{(j)}_{l} = A^{(1)}_{l}$ for any $l \in [n]$ and $j \le k$. 
        \item $j \ge k$ and $j\le i$, We claim that for each such $j$ an exchange will always occur. (The proof is given below.)
          \begin{claim}
            \label{claim:second}
            For any $j \in [k,i]$ we have that in the end of the $j$th iteration:  
            \begin{itemize}
              \item $A^{(j+1)}_{j} = A^{(j)}_{i}$.
              \item $A^{(j+1)}_{i} = A^{(j)}_{j} = A^{(1)}_{j}$.
              \item For any $l > j$ and $l \neq i$ we have $A^{(j+1)}_{l} = A^{(1)}_{l}$.
            \end{itemize}
          \end{claim}

        \item $j > i$, so we know that $A^{(i+1)}_{i}$ is the maximal element in $A$. Therefore, for any $j$, it holds that $A^{(i+1)}_{i}\ge A^{(i)}_{j}$. It follows that no exchange would be made and $A^{(j)}_{i}$ will remain the maximum til the end of the inner loop. Thus for any $j >i$:   
          \begin{equation*}
            \begin{split}
              A^{(j)}_{i}=A^{(j-1)}_{i}=.. =A^{(i+2)}_{i}=A^{(i+1)}_{i}=A^{(i)}_{i-1}=A^{(0)}_{i-1}=\max A
            \end{split}
          \end{equation*}
          And 
          \begin{equation*}
            \begin{split}
             & A^{(j)}_{1}, A^{(j)}_{2}, .. A^{(j)}_{k-1}, A^{(j)}_{k}, A^{(j)}_{k+1}, .. A^{(j)}_{i-1}, A^{(j)}_{i}, A^{(j)}_{i+1} , A^{(j)}_{i+2} , A^{(j)}_{i+3} .. \\
            = & A^{(0)}_{1}, A^{(0)}_{2}, .. A^{(0)}_{k-1}, A^{(0)}_{i}, A^{(0)}_{k}, .. A^{(0)}_{i-2}, A^{(0)}_{i-1}, A^{(0)}_{i+1} , A^{(0)}_{i+2} , A^{(0)}_{i+3} .. 
            \end{split}
          \end{equation*}
          In particular, for $j = n+1$ (Note that there is no $n+1$th iteration). Clearly, the inequalities are satisfied and we are done.
\end{enumerate}
\end{enumerate}
  \end{proof}
  \begin{proof}[Proof of \Cref{claim:second}.]
  Observe that the third section holds trivially by the definition of the algorithm. It doesn't touch any position greater than $j$ in the first $j$ iterations (inner loop) except the $i$th position. So we have to prove only the first two bullets. Again, we are going to prove them by induction on $j$.  
  \begin{enumerate}
    \item Base. $A^{(1)}_{k}$ is greater than $A_{i}$, and by the above case, we have that at the beginning of the $k$th iteration $A^{(k)}_{k}=A^{(1)}_{k}, A^{(k)}_{i}=A^{(1)}_{k}$. Therefore, the condition on line (3) is satisfied, an exchange is made, and $A^{(k+1)}_{k} =A^{(k)}_{i} = A^{(1)}_{i}$ and $A^{(k+1)}_{i} = A^{(k)}_{k}$. %Now, $A^{(k+1)}_{k+1} = A^{(k)}_{k+1} = A^{(0)}_{k+1}$.
    \item Assumption. Assume the correctness of the claim for any $k \le j^{\prime} < j \le i$. 
    \item Step. Consider the $j \in (k,i]$ iteration. By the induction assumption, we have that $A^{(j)}_{j-1} = A^{(j-1)}_{i}$ and $A^{(j)}_{i} = A^{(j-1)}_{j-1} = A^{(1)}_{j-1}$. On the other hand, by the induction assumption of \Cref{claim:first}, $j-1 < i \Rightarrow A^{(1)}_{j-1} \le A^{(1)}_{j}$. Combining the third bullet, we obtain that:                
      \begin{equation*}
        \begin{split}
          A^{(j)}_{j} = A^{(1)}_{j} \ge A^{(1)}_{j-1} = A^{(j)}_{i}
        \end{split}
      \end{equation*}
      And therefore, either there is an inequality and an exchange is made or there is equality. In both cases, after the $j$th iteration, we have $A^{(j+1)}_{j} = A^{(j)}_{i}$ and $A^{(j+1)}_{i} = A^{(j)}_{j} = A^{(1)}_{j}$.
  \end{enumerate}
\end{proof}

\section{Master Theorem.}   
%\author{Master theorem and recursive trees.}
% 
%\begin{abstract}
%    One of the standard methods to analyze the running time of algorithms is to express recursively the number of operations that are made. In the following recitation, we will review the techniques to handle such formulation (solve or bound).  
%\end{abstract}
%
%\begin{algorithm}
%\SetAlgoLined
%\KwResult{returns the multiplication \(x\cdot y\) where \(x,y \in \mathbb{F}^{n}_{2}\) }
% \ \\ 
% \If{ \(x,y \in \mathbb{F}_{2}\) }
%    { return \(x \cdot y\) } 
% \ \\ 
% 
% \Else {
% define \(x_{l} , x_{r} \leftarrow x \) and \(y_{l} , y_{r} \leftarrow x \) \ \ \ \ \ // \( O \left(n\right) \). \\ 
% \ \\ 
% calculate \(z_0 \leftarrow \text{Karatsuba}\left(x_{l},y_{l}\right)\) \\
% \ \ \ \ \ \ \ \ \ \ \ \ \(z_2 \leftarrow \text{Karatsuba}\left(x_{r},y_{r}\right)\) \\ 
% \ \ \ \ \ \ \ \ \ \ \ \ \(z_1 \leftarrow \text{Karatsuba}\left(x_{r} + x_{l} ,y_{l} + y_{r} \right) - z_0 - z_2 \) \\ 
% \ \\
% return \(z_0 + 2^{\frac{n}{2}}z_1 + 2^{n}z_2\) \ \ \ \ \  // \( O \left(n\right) \). 
% }
%\end{algorithm}

% \printbibliography 
\end{document}


