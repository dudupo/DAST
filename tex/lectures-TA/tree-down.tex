
    % Drawing a graph using the PG 3.0 graphdrawing library
% Author: Mark Wibrow
%\documentclass[tikz,border=10pt]{article}
%\usepackage{tikz}
%\usetikzlibrary{positioning}
%\begin{document}

\tikzset{%
  every neuron/.style={
    circle,
    draw,
    %fill=black,
    radius=0.1mm
  },
  neuron missing/.style={
    draw=none,
    scale=2,
    text height=0.25cm,
    execute at begin node=\color{black}$\vdots$
  },
}

\begin{center}

\begin{tikzpicture}%[ultra thick]
    \node [every  neuron ] (v-1) at ( 3 ,-0) { right child  };
    \node [every  neuron ] (v-2) at ( 0 ,-3) { left child  };
    \node [every  neuron ] (v-3) at ( 6 ,-3) {  (original) root };
\node (v-4) at ( 0.0 ,-6) { $ \vdots $  };

\node [every  neuron ] (v-6) at ( 5 ,-6) { $l$  };
\node [every  neuron ] (v-7) at ( 7 ,-6) { $r$  };
\node  (v-8) at ( 5 ,-7) { $\vdots$  };
\node  (v-9) at ( 7 ,-7) { $\vdots$  };
\draw [->] (v-1) -- (v-2);
\draw [->] (v-1) -- (v-3);
\draw [->] (v-2) -- (v-4);
%\draw [->] (v-2) -- (v-5);
\draw [->] (v-3) -- (v-6);
\draw [->] (v-3) -- (v-7);
\draw [->] (v-6) -- (v-8);
\draw [->] (v-7) -- (v-9);
%  node[right=6pt] {$y$} at
\draw[decoration={brace,raise=6pt},decorate, thick]
(8,-1.5) -- node[right=10pt, align=left] {  $n^{\prime}$-size binary tree, \\It's guaranteed that any node, \\beside the root, \\satisfies the heap inequality. } (8, -7.5);

\draw[draw=gray, dashed] (4.5,-7.5) rectangle ++(3.5,6);
\end{tikzpicture} 


\end{center}
