
\documentclass{article}


\usepackage[utf8]{inputenc}
\usepackage[a4paper, total={6in, 9in}]{geometry}
\usepackage{braket}
\usepackage{xcolor}
\usepackage{amsmath}
\usepackage{amsfonts}
\usepackage{tikz}
\usepackage{svg}
\usepackage{graphicx}
\usepackage{media9}
\usepackage{float}
\usetikzlibrary{calc}
\usepackage{array}
\usepackage[ruled,vlined,linesnumbered]{algorithm2e}

\usepackage[
backend=biber,
style=alphabetic,
sorting=ynt
]{biblatex}



\newcommand{\commentt}[1]{\textcolor{blue}{ \textbf{[COMMENT]} #1}}
\newcommand{\ctt}[1]{\commentt{#1}}
\newcommand{\prb}[1]{ \mathbf{Pr} \left[ {#1} \right]}
\newcommand{\onotation}[1]{\(\mathcal{O} \left( {#1}  \right) \)}
\newcommand{\ona}[1]{\onotation{#1}}
\newcommand{\nvar}[2]{ \( #1_{1}, #1_{2} ... #1_{#2}  \) }
%\newenvironment{proof}[0]{\paragraph{Proof.}}{}
%\newenvironment{remark}[0]{\textit{remark}}{}
\newenvironment{cor}[0]{\paragraph{Corollary.}}{}
\newenvironment{example}[0]{\paragraph{Example.}}{}
%\newenvironment{thm}[0]{\paragraph{Theorem.}}{}
\newtheorem{prop}{Proposition}
\newtheorem{ex}{Exercise}
\newtheorem{sol}{Solution}
\newtheorem{theorem}{Theorem} 		
\newtheorem{thm}{Theorem}[section]
\newtheorem{conj}[thm]{Conjecture} 	
\newtheorem{lemma}[thm]{Lemma}
\newtheorem{corollary}[thm]{Corollary} 
\newtheorem{claim}[thm]{Claim}
\newtheorem{proposition}[thm]{Proposition}
\newtheorem{definition}{Definition} 
\newtheorem{remark}{Remark}
 


% \addbibresource{sample.bib} %Import the bibliography file
\begin{document}    
\maketitle
\begin{document}
\ifdefined\BOOK
\else
\setcounter{chapter}{5}
\fi
\chapter{Binary Search Trees.} 

\section{Binaries Trees and How to Encode Them.} We have already seen, in heaps, that organizing our data in a graph-like structure can offer a speed advantage. For future applications, and in particular for maintaining data in sorted order, we will have to encode our data using binary trees. These trees may not be almost complete and also have to support pointer manipulations, specifically placing a binary tree as a left or right subtree of a given node. To enable this, we will have to treat the \textbf{right}, \textbf{left}, and \textbf{parent} as variables, in contrast to heaps where they are determined completely by the node index. We begin this section by stating definitions.


\begin{definition}
  \begin{enumerate}
    \item Binary Tree: A tree in which any vertex has at most two children.
    \item A descendant of vertex $x$ is a vertex in the subtree whose root is $x$. A left descendant of vertex $x$ is either a vertex in the subtree whose root is the left child of $x$, or $x$'s left child.
    \item An ancestor of $x$ is a vertex to which $x$ belongs as a descendant.
    \item A leaf is a vertex without children.
    \item Height of vertex $x$ is the length of the longest simple path (without cycles) between $x$ and one of the leaves.
    \item Height of the tree is the height of its root, which is usually denoted by $h$.
\end{enumerate}
\end{definition}

We encode a binary tree by associating a field to each vertex $x$, representing its right, left children, and parent. We use the notation $x$.left to refer to the left child of $x$, although the physical implementation may differ conceptually. For example, the way binary trees are implemented in Cormen is through 4 arrays. The first stores the value of $x$, while the others store pointers of specific types. For instance, the array LEFT, where LEFT.$x$ stores the left child of $x$.


If nothing else has been mentioned, then we can assume that we can add additional fields to the vertices.


\section{Binary Search Trees.} Binary search tree (BST) is a binary tree which any node $x$ of it: 
\begin{enumerate}
  \item Contains a field key, storing a number $x$.key. 
  \item Any left descendant $y$ of $x$ satisfies $y$.key $\le$ $x$.key. 
  \item Any right descendant $y$ of $x$ satisfies $y$.key $\ge$ $x$.key. 
\end{enumerate}

\paragraph{Question.} Let $T$ be a binary search tree, Where are the minimum and maximum values of $T$? (most left and right nodes). 

\begin{definition}
Let $T$ be a binary search tree, and let $x$ be a node belonging to it. The predecessor of $x$ will be defined as a vertex $y$ such that $y$.key $\leq x$.key and $y$.key is maximal among the nodes satisfying this condition. If we were to set the values of $T$ in sorted order, then the predecessor of $x$ would be located on its left. The successor of $x$ will be defined as $y$, where $x$ is the predecessor of $y$.
\end{definition}


%We are about to present another $\Theta(n^{2})$-sorting algorithm, whose correctness is not so obvious. The algorithm was developed by Stanley P. Y. Fung, \cite{Simplesort}, who coined its name - "ICan'tBelieveItCanSort" - due to the surprise of having such a simple sorting algorithm. It's worth mentioning that, despite its simplicity, Fung came up with this algorithm in 2021.
%
%
%\begin{algorithm}
%\SetAlgoLined
%\KwResult{Sorting $A_{1},A_{2},..A_{n}$ }
%\caption{ "ICan'tBelieveItCanSort"  alg.}
%\For{ $ i \in [n]$} {
%  \For{ $ j \in [n]$} {
%    \If { $A_{i} < A_{j} $} {
%      swap $A_{i} \leftrightarrow A_{j}$
%    }
%  }
%}
%
%
%\end{algorithm}


% \printbibliography 
\end{document}


