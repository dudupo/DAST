\title{Heaps - Recitation 4} 
\author{Correctness, Loop Invariants And Heaps.}
\documentclass{article}


\usepackage[utf8]{inputenc}
\usepackage[a4paper, total={6in, 9in}]{geometry}
\usepackage{braket}
\usepackage{xcolor}
\usepackage{amsmath}
\usepackage{amsfonts}
\usepackage{tikz}
\usepackage{svg}
\usepackage{graphicx}
\usepackage{media9}
\usepackage{float}
\usetikzlibrary{calc}
\usepackage{array}
\usepackage[ruled,vlined,linesnumbered]{algorithm2e}

\usepackage[
backend=biber,
style=alphabetic,
sorting=ynt
]{biblatex}



\newcommand{\commentt}[1]{\textcolor{blue}{ \textbf{[COMMENT]} #1}}
\newcommand{\ctt}[1]{\commentt{#1}}
\newcommand{\prb}[1]{ \mathbf{Pr} \left[ {#1} \right]}
\newcommand{\onotation}[1]{\(\mathcal{O} \left( {#1}  \right) \)}
\newcommand{\ona}[1]{\onotation{#1}}
\newcommand{\nvar}[2]{ \( #1_{1}, #1_{2} ... #1_{#2}  \) }
%\newenvironment{proof}[0]{\paragraph{Proof.}}{}
%\newenvironment{remark}[0]{\textit{remark}}{}
\newenvironment{cor}[0]{\paragraph{Corollary.}}{}
\newenvironment{example}[0]{\paragraph{Example.}}{}
%\newenvironment{thm}[0]{\paragraph{Theorem.}}{}
\newtheorem{prop}{Proposition}
\newtheorem{ex}{Exercise}
\newtheorem{sol}{Solution}
\newtheorem{theorem}{Theorem} 		
\newtheorem{thm}{Theorem}[section]
\newtheorem{conj}[thm]{Conjecture} 	
\newtheorem{lemma}[thm]{Lemma}
\newtheorem{corollary}[thm]{Corollary} 
\newtheorem{claim}[thm]{Claim}
\newtheorem{proposition}[thm]{Proposition}
\newtheorem{definition}{Definition} 
\newtheorem{remark}{Remark}
 


% \addbibresource{sample.bib} %Import the bibliography file
\begin{document}    
\maketitle

%\begin{abstract}
  Apart of quantify the resource requirement of our algorithms we are also intersted to prove that they indeed work. In this Recitation we will demonstrate how to prove correctness via the notation of loop invariant. In addition we will present the first (non-trival) data structre in course, and prove that it allow us to compute the maximum efficintly.     
   
%\end{abstract}


\subsection*{Correctness And Loop Invariant.}
In this course, any algorithm is defined relative to a task/problem/function, And it will be said correct if for any input it compute diserable output. For example, suppose that our task is to extract the maximum element from a given array.
So the input space are all the arrays of numbers, and proving that a given algorithm is correct, requires to prove that for an aribtrary array the algorithm's out put is the the maximal number. Formally:  

\begin{defbox}{Correctness.}
We will say that an algorithm \( \mathcal{A}\) (an ordered set of operations) computes \( f:D_1 \rightarrow D_2 \) if for every \(x \in D_1 \Rightarrow f(x) = \mathcal{A}(x)\). Sometimes when it's obvious what is the goal function \(f\), we will abbreviate and say that \( \mathcal{A}\) is correct.       
\end{defbox}
\paragraph{}
Other Examples of functions \(f\) might be including any compution taks: file saving, summing numbers, posting a message in the forum, etc. Let's dive back into the maximum extraction problem and see how correctenss should be prove in practice.     
\paragraph{Task: Maximum Finding.} \textit{Given $n\in \mathbb{N}$ numbers $a_1, a_2, \cdots a_n \in \mathbb{R}$ write an Algorithm which returns their maximum.} 

Consider the follows suggestion. How would you prove it correct?  
\begin{algbox}{Maximum finding.}
\begin{algorithm*}[H]
% \SetAlgoLined
\KwResult{returns the maximum of \(a_1 ... a_n \in \mathbb{R}^n \)  }
 let \(b \leftarrow a_1 \) \\ 
 \For{\(i \in [2, n] \) } { 
        \(b \leftarrow \max \left(b, a_i \right) \)
    } 
 return \( b \) 
 %\caption{maximum alg.}
\end{algorithm*}
\end{algbox}

Usually it will be convinant to divide the algortims into subsections and then characteristic, and prove correctness for each of them seapratly. One main technique is usning the notation of Loop Invariants. Loop Invariant is a property that characteristic a loop segment code  and satisfy the following conditions: 
\begin{defbox}{Loop Invariant.} 
\begin{enumerate}
    \item Initialization. The property holds (even) before the first iteration of the loop.   
    \item Conservation. As long as one performs the loop iterations, the property still holds.
    \item (optional) Termination. Exiting from the loop carrying information.
\end{enumerate}
\end{defbox}


What is the Loop Invariant here? \textit{"at the \(i\)-th iteration, \(b = \max{ \{ a_1 ... a_{i-1} \} } \)"}. The proof is almost identical to the naive case.   

\paragraph{Claim.} Consider the while loop. The property: \textit{"for every \(j^\prime < j \le n+1 \Rightarrow a_{j^\prime} \le a_i \)"} is a loop invariant that is associated with it. 

\textbf{Proof:} first, the initialization condition holds, as the at the first iteration \(j=1\) and therefore the property is trivial.
Assume by induction, that for every \(m < j\) the property is correct, and consider the \(j\)-th iteration. If back again to line (5), then it means that \( (j-1) < n\) and \( a_{j-1} \le a_{i} \). Combining the above with the induction assumption yields that \(a_i \ge a_{j-1},a_{j-2}, ... a_{1}\).    

\paragraph{Correctness Proof.} Split into cases, First if the algorithm return result at line (9), then due to the loop invariant, combining the fact that \( j = n + 1\), it holds that for every \(j^\prime  \le n < j \Rightarrow a_i \ge a_{j^\prime} \)  i.e \(a_i\) is the maximum of \(a_1, .... a_n \). The second case, in which the algorithm returns \( \Delta \) at line number (10) contradicts the fact that \(n\) is finite, and left as an exercise.  the running time is \( O(n^2) \) and the space consumption is \(O(n)\). 

\paragraph{Task: Element finding.}  \textit{Given $n\in \mathbb{N}$ numbers $a_1, a_2, \cdots a_n \in \mathbb{R}$ and additional number $x \in \mathbb{R}$ write an Algorithm that returns $i$ s.t $a_{i} = x$ if there exists such $i$ and} False \textit{otherwise.} 

\begin{algbox}{Element finding.}
\begin{algorithm}[H]
% \SetAlgoLined
\KwResult{returns the maximum of \(a_1 ... a_n \in \mathbb{R}^n \)  }
 \ \\ 
 \For{ \(i \in [n] \) } { 
        \ \\ 
	\If { \(a_{i} = x\) }{
	  return \(i, a_{i}\)
        }
    } 
    return \( \Delta \) 
\end{algorithm}
\end{algbox}

\paragraph{Loop Invariant In The Cleverer Alg.} Consider now the linear time algorithm:

\subsection*{Heaps.} We have seen in the Lecture that no Algorithm can compute the $\max$ function with less then $n-1$ comparisions. So our soultion above is indeed the best we could excpect for,Yet it clears that if we 



% \printbibliography 
\end{document}

