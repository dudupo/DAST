
\documentclass{article}


\usepackage[utf8]{inputenc}
\usepackage[a4paper, total={6in, 9in}]{geometry}
\usepackage{braket}
\usepackage{xcolor}
\usepackage{amsmath}
\usepackage{amsfonts}
\usepackage{tikz}
\usepackage{svg}
\usepackage{graphicx}
\usepackage{media9}
\usepackage{float}
\usetikzlibrary{calc}
\usepackage{array}
\usepackage[ruled,vlined,linesnumbered]{algorithm2e}

\usepackage[
backend=biber,
style=alphabetic,
sorting=ynt
]{biblatex}



\newcommand{\commentt}[1]{\textcolor{blue}{ \textbf{[COMMENT]} #1}}
\newcommand{\ctt}[1]{\commentt{#1}}
\newcommand{\prb}[1]{ \mathbf{Pr} \left[ {#1} \right]}
\newcommand{\onotation}[1]{\(\mathcal{O} \left( {#1}  \right) \)}
\newcommand{\ona}[1]{\onotation{#1}}
\newcommand{\nvar}[2]{ \( #1_{1}, #1_{2} ... #1_{#2}  \) }
%\newenvironment{proof}[0]{\paragraph{Proof.}}{}
%\newenvironment{remark}[0]{\textit{remark}}{}
\newenvironment{cor}[0]{\paragraph{Corollary.}}{}
\newenvironment{example}[0]{\paragraph{Example.}}{}
%\newenvironment{thm}[0]{\paragraph{Theorem.}}{}
\newtheorem{prop}{Proposition}
\newtheorem{ex}{Exercise}
\newtheorem{sol}{Solution}
\newtheorem{theorem}{Theorem} 		
\newtheorem{thm}{Theorem}[section]
\newtheorem{conj}[thm]{Conjecture} 	
\newtheorem{lemma}[thm]{Lemma}
\newtheorem{corollary}[thm]{Corollary} 
\newtheorem{claim}[thm]{Claim}
\newtheorem{proposition}[thm]{Proposition}
\newtheorem{definition}{Definition} 
\newtheorem{remark}{Remark}
 


% \addbibresource{sample.bib} %Import the bibliography file
\begin{document}    
\maketitle
\begin{document}
\ifdefined\BOOK
\else
\setcounter{chapter}{10}
\fi
\chapter{Minimum Spanning Tree Recitation.} 


\section{The MST Problem.}

\begin{definition}
  A spanning tree $T$ of a graph $G=(V,E)$ is a subset of edges in $E$ such that $T$ is a tree (having no cycles), and the graph $(V,T)$ is connected.   
\end{definition}

\begin{problem}[MST] Let $G = (V,E)$ be a weighted graph with weight function $w : E \rightarrow \mathbb{R}$. We extend the weight function to subsets of $E$ by defining the weight of $X\subset E$ to be $w(X)= \sum_{e \in X}w(e)$. The minimum spanning tree (MST) of $G$ is the spanning tree of $G$ that has the minimal weight according to $w$. Note that in general, there might be more than one MST for $G$. 
\end{problem}

\begin{definition}
  \label{def:cut}
  Let $U \subset V$. We define the cut associated with $U$ as the set of outer edges of $U$, namely all the edges $(u,v)\in E$ such that $u\in U$ and $v \notin U$. We use the notation $X = (U, \bar{U})$ to represent the cut. We say that $E^{\prime} \subset E$ respects the cut if $E^{\prime} \cap X = \emptyset$.
\end{definition} 

\begin{lemma}[The Cut-Lemma] 
  \label{lemma:cut}
  Let $T$ be an MST of $G$. Consider a forest $F \subset T$ and a cut $X$ that respects $X$ (i.e. $F \cap X = \emptyset$). Then $F \cup \text{ arg}\min_{e} w(e)$ is also contained in some MST. Note that it does not necessarily have to be the same tree $T$. 
\end{lemma}

\begin{proof}
If $e \in T$, then $F \cup \{ e \} \subset T$ and we are done. Otherwise, consider the second case where $e \notin T$. This means that $T \cup \{ e \}$ has $|V|$ edges and therefore must have a cycle. Let $\Gamma = T \cup \{ e \}$ and let $x$ and $y$ be the endpoints of $e$ (namely $e = (x,y)$). Denote the subset of vertices defining the cut $X$ by $U$. Without loss of generality, let's assume $x \in U$ and $y \in \bar{U}$.


Since $T$ is connected, there is a path $x \rightsquigarrow y$ in $T$, denote it by $\mathcal{P}$. Additionally, because $e \notin T$, we have that $e \notin \mathcal{P}$. This means that there must be another edge in $\mathcal{P}$ connecting a vertex in $U$ to a vertex in $\bar{U}$\footnote{Otherwise, walking along $\mathcal{P}$ cannot take one out of $U$, leading to a contradiction as $\mathcal{P}$ leads to $y$.}. Let $e^{\prime}$ be that edge, we have:\begin{enumerate}
    \item Both $e^{\prime}, e \in X$ So $w(e) \le w(e^{\prime})$.
    \item $e \cup \mathcal{P}$ is a cycle in $\Gamma$. 
  \end{enumerate}

By using the fact that subtracting an edge from a cycle doesn't harm connectivity (see \Cref{claim:subtract}), we can conclude that $\Gamma/\{e^{\prime}\}$ is connected. Since it has $|V|-1$ edges, it must be a spanning tree. On the other hand, by:
  \begin{equation*}
    \begin{split}
      w\left( \Gamma/\{e^{\prime}\} \right) = w\left( T \right) + \overbrace{w(e) - w(e^{\prime})}^{ \le 0} \le w\left( T \right) 
    \end{split}
  \end{equation*}
  So $\Gamma/\{e^{\prime}\}$ is an MST.  Finally, to close the proof, observe that $F \cup \{e\} \subset \Gamma/\{e^{\prime}\}$. This means that, we have found an MST that contains $F \cup \{e\}$. 

\end{proof}
\section{Kruskal Algorithm.} This algorithm constructs the MST iteratively by holding a forest $F$ contained in an MST and then looking for the minimal edge in a cut that it respects. Note, that since $F$ has no cycles, any edge $e \subset E$ that does not create a cycle in $F$ must belong to a cut $X$ that is respected by $F$. By ensuring that the edges are examined in increasing weight order, we can determine that the first edge that does not create a cycle is also the one with the minimum weight among them. Therefore, according to \Cref{lemma:cut}, we can conclude that the forest obtained by adding $e$ into $F$ is contained in an MST.

\begin{algorithm} 
\SetAlgoLined
\KwResult{Returns MST of given $G=(V,E, w)$ }
\caption{ Kruskal alg.}
sorts the $E$ according to $w$ \\
define $F_{0} = \emptyset$ and $i \leftarrow 0$ \\
\For{ $ e \in E$ in sorted order } {
  \If { $F_{i} \cup \{e\}$ has no cycle} {
    $F_{i+1} \leftarrow F_{i} \cup \{e\}$ \\
    $i \leftarrow i + 1$
  }
}
\Return $F_{i}$
  \label{alg:krus}
\end{algorithm}

\begin{claim}
  For any $i$, $F_{i} \subset $ of an MST.
\end{claim}
\begin{proof}
  By induction. 
  \begin{enumerate}
    \item Base. Let $T$ be an arbitrary MST of $G$. $F_{0} = \emptyset \subset T $.
    \item Assumption. Assume correctness for any $j < i$. 
    \item Step. By the induction assumption there is an MST $T$ such $F_{i-1} \subset T$. Denote by $e$ the edge for which $F_{i} = F_{i-1}\cup \{e\}$. By algorithm definition, $F_{i} = F_{i-1} \cup \{e\}$ has no cycles (line number (4)).   
  \end{enumerate}
\end{proof}

\begin{proof}[Correctness of \Cref{alg:krus}]
 
\end{proof}



\begin{claim}
  \label{claim:subtract}
  Let $G$ be a connected graph containing a cycle $C$. Then the subtraction of any an edge in $C$ gives a connected graph. 
\end{claim}
\begin{proof}
Assume, by contradiction, that a graph $G^{\prime} = G / \{ e \} $, where $e \in C$, is not connected. This means that there are two vertices $u$ and $v$ that have a path between them in $G$, but no such path exists in $G^{\prime}$. Denote this path by $\mathcal{P}$ and observe that $e \in \mathcal{P}$, otherwise, $\mathcal{P}$ would also be a path from $u$ to $v$ in $G^{\prime}$.

Denote the ends of $e$ by $(x,y)=e$. Also, denote $C$ by $\braket{x_{0},x_{1},..x_{i},x,y,y_{0},..,y_{j}}$, where $y_{j}=x_{0}$ and there is an inequality for any other pair of vertices (we used the cycle definition). Then, there is a path $x \rightsquigarrow y$ in $C$, defined by 
\begin{equation*}
  \begin{split}
\braket{x_{i},x_{i-1},..,x_{1},x_{0},y_{j-1},y_{j-2},..,y_{0},y}
  \end{split}
\end{equation*}
 We denote this path by $\mathcal{P}^{\prime}$. By replacing $e$ in $\mathcal{P}$ with $\mathcal{P}^{\prime}$, we obtain a path $u \rightsquigarrow x \rightsquigarrow^{\mathcal{P}^\prime} y \rightsquigarrow v$, which is a path between $u$ and $v$ that does not contain $e$. This contradicts the assumption that there is no path between $u$ and $v$ in $G^{\prime}$.
\end{proof}




% \printbibliography 
\end{document}


