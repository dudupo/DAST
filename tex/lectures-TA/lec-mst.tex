
\documentclass{article}


\usepackage[utf8]{inputenc}
\usepackage[a4paper, total={6in, 9in}]{geometry}
\usepackage{braket}
\usepackage{xcolor}
\usepackage{amsmath}
\usepackage{amsfonts}
\usepackage{tikz}
\usepackage{svg}
\usepackage{graphicx}
\usepackage{media9}
\usepackage{float}
\usetikzlibrary{calc}
\usepackage{array}
\usepackage[ruled,vlined,linesnumbered]{algorithm2e}

\usepackage[
backend=biber,
style=alphabetic,
sorting=ynt
]{biblatex}



\newcommand{\commentt}[1]{\textcolor{blue}{ \textbf{[COMMENT]} #1}}
\newcommand{\ctt}[1]{\commentt{#1}}
\newcommand{\prb}[1]{ \mathbf{Pr} \left[ {#1} \right]}
\newcommand{\onotation}[1]{\(\mathcal{O} \left( {#1}  \right) \)}
\newcommand{\ona}[1]{\onotation{#1}}
\newcommand{\nvar}[2]{ \( #1_{1}, #1_{2} ... #1_{#2}  \) }
%\newenvironment{proof}[0]{\paragraph{Proof.}}{}
%\newenvironment{remark}[0]{\textit{remark}}{}
\newenvironment{cor}[0]{\paragraph{Corollary.}}{}
\newenvironment{example}[0]{\paragraph{Example.}}{}
%\newenvironment{thm}[0]{\paragraph{Theorem.}}{}
\newtheorem{prop}{Proposition}
\newtheorem{ex}{Exercise}
\newtheorem{sol}{Solution}
\newtheorem{theorem}{Theorem} 		
\newtheorem{thm}{Theorem}[section]
\newtheorem{conj}[thm]{Conjecture} 	
\newtheorem{lemma}[thm]{Lemma}
\newtheorem{corollary}[thm]{Corollary} 
\newtheorem{claim}[thm]{Claim}
\newtheorem{proposition}[thm]{Proposition}
\newtheorem{definition}{Definition} 
\newtheorem{remark}{Remark}
 


% \addbibresource{sample.bib} %Import the bibliography file
\begin{document}    
\maketitle

\begin{document}
\ifdefined\BOOK
\else
\setcounter{chapter}{10}
\fi
\chapter{Minimum Spanning Tree Recitation.} 


\section{The Spanning Tree Problem.}


\begin{definition}
  A spanning tree $T$ of a graph $G=(V,E)$ is an edges subset of $E$ such that $T$ is a tree (having no cycles), and the graph $(V,T)$ is connected.   
\end{definition}

\begin{problem}[MST] Let $G = (V,E)$ be a weight graph with weight function $w : E \rightarrow \mathbb{R}$. Let's extends the weight for $E$'s subsets by defining for the weight of $X\subset E$ to be $w(X)= \sum_{e \in X}w(e)$. The minimum spanning tree of $G$ is the spanning tree of $G$ that has the minimal weight according to $w$.
\end{problem}

\begin{definition}
  Let $U \subset V$ we will define the cut associated by $U$ as the outer edges of $U$ we use the following $C = (U, \bar{U})$ to denote the cut. 
\end{definition} 
\input{../texlib/tail}


