
    % Drawing a graph using the PG 3.0 graphdrawing library
% Author: Mark Wibrow
%\documentclass[tikz,border=10pt]{article}
%\usepackage{tikz}
%\usetikzlibrary{positioning}
%\begin{document}

\tikzset{%
  every neuron/.style={
    circle,
    draw,
    %fill=black,
    radius=0.1mm
  },
  neuron missing/.style={
    draw=none,
    scale=2,
    text height=0.25cm,
    execute at begin node=\color{black}$\vdots$
  },
}

\begin{tikzpicture}
    \node [every  neuron ] (v-1) at ( 0.0 ,-0) { $2$  };
  \node [every  neuron ] (v-2) at ( 0.0 ,-1) { $4$  };
\node [every  neuron ] (v-3) at ( 1.5 ,-1) { $2$  };
\node [every  neuron ] (v-4) at ( 0.0 ,-2) { $5$  };
\node [every  neuron ] (v-5) at ( 0.75 ,-2) { $6$  };
\node [every  neuron ] (v-6) at ( 1.5 ,-2) { $3$  };
\draw [->] (v-1) -- (v-2);
\draw [->] (v-1) -- (v-3);
\draw [->] (v-2) -- (v-4);
\draw [->] (v-2) -- (v-5);
\draw [->] (v-3) -- (v-6);
\end{tikzpicture}
