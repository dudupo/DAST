\documentclass{article}


\usepackage[utf8]{inputenc}
\usepackage[a4paper, total={6in, 9in}]{geometry}
\usepackage{braket}
\usepackage{xcolor}
\usepackage{amsmath}
\usepackage{amsfonts}
\usepackage{tikz}
\usepackage{svg}
\usepackage{graphicx}
\usepackage{media9}
\usepackage{float}
\usetikzlibrary{calc}
\usepackage{array}
\usepackage[ruled,vlined,linesnumbered]{algorithm2e}

\usepackage[
backend=biber,
style=alphabetic,
sorting=ynt
]{biblatex}



\newcommand{\commentt}[1]{\textcolor{blue}{ \textbf{[COMMENT]} #1}}
\newcommand{\ctt}[1]{\commentt{#1}}
\newcommand{\prb}[1]{ \mathbf{Pr} \left[ {#1} \right]}
\newcommand{\onotation}[1]{\(\mathcal{O} \left( {#1}  \right) \)}
\newcommand{\ona}[1]{\onotation{#1}}
\newcommand{\nvar}[2]{ \( #1_{1}, #1_{2} ... #1_{#2}  \) }
%\newenvironment{proof}[0]{\paragraph{Proof.}}{}
%\newenvironment{remark}[0]{\textit{remark}}{}
\newenvironment{cor}[0]{\paragraph{Corollary.}}{}
\newenvironment{example}[0]{\paragraph{Example.}}{}
%\newenvironment{thm}[0]{\paragraph{Theorem.}}{}
\newtheorem{prop}{Proposition}
\newtheorem{ex}{Exercise}
\newtheorem{sol}{Solution}
\newtheorem{theorem}{Theorem} 		
\newtheorem{thm}{Theorem}[section]
\newtheorem{conj}[thm]{Conjecture} 	
\newtheorem{lemma}[thm]{Lemma}
\newtheorem{corollary}[thm]{Corollary} 
\newtheorem{claim}[thm]{Claim}
\newtheorem{proposition}[thm]{Proposition}
\newtheorem{definition}{Definition} 
\newtheorem{remark}{Remark}
 


% \addbibresource{sample.bib} %Import the bibliography file
\begin{document}    
\maketitle

\section{Correctness - Recitation 2} 
\author{Correctness proofs and computational complexity. }

 
\begin{paragraph}
    Proving algorithms correctness is necessary to guarantee that our code computes its goal for every given input. In that recitation, we will examine several algorithms, analyze theirs running time and memory consumption, and prove they are correct.   
\end{paragraph}


\paragraph{Leading Example.}
Consider \(n\) numbers \(a_1,a_2,....,a_n \in \mathbb{R}\). Given set \(Q\) of \(|Q|\) queries, such each query \(q \in Q\) is a tuple \( (i,j) \in [n] \times [n] \). Write an algorithm that calculates the \(\max_{i\le k\le j}{a_k} \). 

\section{Correctness And Loop Invariant.}

\paragraph{Correctness.} We will say that an algorithm \( \mathcal{A}\) (an ordered set of operations) computes \( f:D_1 \rightarrow D_2 \) if for every \(x \in D_1\) the following equality holds \(f(x) = \mathcal{A}(x)\). Sometimes when it's obvious what is the goal function \(f\), we will abbreviate and say that \( \mathcal{A}\) is correct.       

Examples of functions \(f\) might be: file saving, summing numbers, or posting a message in the forum.  

\paragraph{Loop Invariant.} Loop Invariant is a property that characteristic a loop segment code  and satisfy the following conditions: 
\begin{enumerate}
    \item Initialization. The property holds (even) before the first iteration of the loop.   
    \item Conservation. As long as one performs the loop iterations, the property still holds.
    \item (optional) Termination. Exiting from the loop carrying information.
\end{enumerate}

\paragraph{Example.} Before dealing with the hard problem, let us face the naive algorithm to find the maximum of a given array.

\begin{algorithm}[H]
% \SetAlgoLined
\KwResult{returns the maximum of \(a_1 ... a_n \in \mathbb{R}^n \)  }
 \ \\ 
 \For{ \(i \in [n] \) } { 
        \( j \leftarrow 1 \) \\
        \ \\
        \While{ \(j \le [n] \) and \( a_i \ge a_j \) } {
        \( j \leftarrow j + 1 \)    
        }
        \\
        \ \\ 
        \If { \(j = n+1\) }{
        return \(a_i\)
        }
    } 
return \( \Delta \) 
 \caption{naive maximum alg.}
\end{algorithm}
\paragraph{Claim.} Consider the while loop. The property: \textit{"for every \(j^\prime < j \le n+1 \Rightarrow a_{j^\prime} \le a_i \)"} is a loop invariant that is associated with it. 

\textbf{Proof:} first, the initialization condition holds, as the at the first iteration \(j=1\) and therefore the property is trivial.
Assume by induction, that for every \(m < j\) the property is correct, and consider the \(j\)-th iteration. If back again to line (5), then it means that \( (j-1) < n\) and \( a_{j-1} \le a_{i} \). Combining the above with the induction assumption yields that \(a_i \ge a_{j-1},a_{j-2}, ... a_{1}\).    

\paragraph{Correctness Proof.} Split into cases, First if the algorithm return result at line (9), then due to the loop invariant, combining the fact that \( j = n + 1\), it holds that for every \(j^\prime  \le n < j \Rightarrow a_i \ge a_{j^\prime} \)  i.e \(a_i\) is the maximum of \(a_1, .... a_n \). The second case, in which the algorithm returns \( \Delta \) at line number (10) contradicts the fact that \(n\) is finite, and left as an exercise.  the running time is \( O(n^2) \) and the space consumption is \(O(n)\). 

\paragraph{Loop Invariant In The Cleverer Alg.} Consider now the linear time algorithm:

\begin{algorithm}[H]
% \SetAlgoLined
\KwResult{returns the maximum of \(a_1 ... a_n \in \mathbb{R}^n \)  }
 \ \\ 
 let \(b \leftarrow a_1 \) \\ 
 \ \\ 
 \For{\(i \in [2, n] \) } { 
        \(b \leftarrow \max \left(b, a_i \right) \)
    } 
 return \( b \) 
 \caption{maximum alg.}
\end{algorithm}

What is the Loop Invariant here? \textit{"at the \(i\)-th iteration, \(b = \max{ \{ a_1 ... a_{i-1} \} } \)"}. The proof is almost identical to the naive case.   

\section{Non-Linear Space Complexity Algorithms. }
\paragraph{Sub-Array Maximum.} Consider the leading example; It's easy to write an algorithm that answers the queries at a total time of a \( O\left( |Q| \cdot n \right) \) by answers separately on each query. Can we achieve a better upper bound?

\begin{algbox}{Sub-Array. \(O(n^2)\) space alg.}
\begin{algorithm}[H]
% \SetAlgoLined
\KwResult{print the \( \max{\{ a_i ... a_j \} }\) for each query \((i,j) \in Q \) }
 \ \\ 
 let \(A \leftarrow \mathbb{M}^{n\times n} \) \\ 
 \ \\ 
 \For{\(i \in [n] \) } {
    \( A_{i,i} \leftarrow a_i\)
 }
 \ \\
 \For{ \(k \in [1, n]\) }{
    \For{\(i \in [n] \) } {
        \If{ \(i+k \le n\) }{
        \(A_{i,i+k} \leftarrow \max \left(A_{i,i+k-1}, a_{i+k} \right) \)
        }
    } 
}
\ \\
\For { \( q \in Q \) }{
    \(i,j \leftarrow q \) \\
    print \( A_{i,j}\)
}
\end{algorithm}
\end{algbox}

\paragraph{Claim.} Consider the outer loop at the \(k\)-th step. The following is a loop invariant: \[for \ every \ k^\prime < k ,\ s.t \ i + k^\prime \le n \Rightarrow A_{i,i+k^\prime} = \max{ \{ a_{i}, a_{i+1}, ... ,a_{i + k^\prime} \} }\]  
\textbf{Proof.} The initialization condition trivially holds, assume by induction that \( A_{i,i+k-1} = \max{\{a_i ... a_{i+k-1}\}}\) at beginning of \( k \) iteration. By the fact that \( \max(x,y,z)= \max(\max(x,y),z) \) we get that
\begin{equation*}
\begin{split}
 \max{\{a_1 ... a_{i + k-1}, a_{i+ k} \}} = \max{\{ \max{ \{ a_1 ... a_{i + k-1} \} }, a_{i+ k} \}} =  \max{\{A_{i,i+k-1}, a_{i+ k} \}}
 \end{split}    
 \end{equation*} And the right term is exactly the value which assigned to \(A_{i,i+k}\) in the end of the\(k\)-th iteration. Thus in the beginning of \( k+1 \) iteration the property is still conserved.

\paragraph{ \(O\left(n\log n\right)\) Space Solution.} Example for \(O\left(n\log n + |Q|\log n\right)\) time and \(O\left(n\log n\right)\) space algorithm. Instead of storing the whole matrix, we store only logarithmic number of rows.   

\begin{algbox}{Sub-Array. \(O(n \log n )\) space alg.}
\begin{algorithm}[H]
% \SetAlgoLined
\KwResult{print the \( \max{\{ a_i ... a_j \} }\) for each query \((i,j) \in Q \) }
 \ \\ 
 let \(A \leftarrow \mathbb{M}^{n\times \log n} \) \\ 
 \ \\
 \For{\(i \in [n] \) } {
    \( A_{i,1} \leftarrow a_i\)
 }
 \ \\
 \For{ \(k \in [2,4,..,2^m,...,n]\) }{
    \For{\(i \in [n] \) } {
        \If{ \(i+k \le n\) }{
        \(A_{i,k} \leftarrow \max \left(A_{i,\frac{k}{2}},A_{i+\frac{k}{2}, \frac{k}{2}} \right) \)
        }
    } 
}
\ \\
\For { \( q \in Q \) }{
    \(i,j \leftarrow q \) \\
    decompose \(j - i \) into binary representation \(2^{t_1} + 2^{t_2} + .. +2^{t_l}\) \\
    print \( \max { \{ A_{i,2^{t_1}},A_{i+ 2^{t_1}, 2^{t_2} }, ... , A_{i+ 2^{t_1} + 2^{t_2} +.. 2^{t_{l-1}}, 2^{t_l}} \} }\)
}
\end{algorithm}
\end{algbox}




\section{Recursive Analysis - Recitation 3} 
\author{Master theorem and recursive trees.}

 
\begin{paragraph}
    One of the standard methods to analyze the running time of algorithms is to express recursively the number of operations that are made. In the following recitation, we will review the techniques to handle such formulation (solve or bound).  
\end{paragraph}


\section{Bounding recursive functions by hands.} Our primary tool to handle recursive relation is the Master Theorem, which was proved in the lecture. As we would like to have a more solid grasp, let's return on the calculation in the proof over a specific case. 
Assume that your algorithm analysis has brought the following recursive relation:
\begin{enumerate}
    \item \textbf{Example.} \( T\left(n\right)  = \left\{ \begin{array}{rcl}
& 4T\left(\frac{n}{2}\right)+c\cdot n & \mbox{for }  n > 1  \\
& 1 & \mbox{else}  
\end{array}\right. \). Thus, the running time is given by \begin{equation*}
    \begin{split}
 & T\left(n\right)  = 4T\left(\frac{n}{2}\right)+c\cdot n=  4\cdot4T\left(\frac{n}{4}\right)+4c\cdot\frac{n}{2}+c\cdot n = ... = \\ & \overset{\text{\textcolor{red}{critical}}}{\overbrace{4^{h}T(1)}} + c\cdot n\left(1+\frac{4}{2}+\left(\frac{4}{2}\right)^{2}...+\left(\frac{4}{2}\right)^{h-1}\right) = 4^{h} + c\cdot n\cdot\frac{2^{h}-1}{2-1}
    \end{split}
\end{equation*}
We will call the number of iteration till the stopping condition the recursion height, and we will denote it by \(h\) . What should be the recursion height? \( 2^{h} = n \Rightarrow h =\log\left(n\right) \). So in total we get that the algorithm running time equals \( \Theta\left(n^2\right)\). 

\textbf{Question}, Why is the term \( 4^{h} T(1) \) so critical? Consider the case \(T\left(n\right) =  4T\left(\frac{n}{2}\right) + c \) .One popular mistake is to forget the final term, which yields a linear solution \( \Theta(n)\) (instead of quadric \( \Theta(n^2)\)).   

    \item \textbf{Example.} \( T\left(n\right) & = \left\{ \begin{array}{rcl}
& 3T\left(\frac{n}{2}\right) + c\cdot n & \mbox{for }  n > 1  \\
& 1 & \mbox{else}  
\end{array}\right. \), and then the expanding yields: 
\begin{equation*}
    \begin{split}
        T\left(n\right) & = 3T\left(\frac{n}{2}\right) + c\cdot n = 3^2 T\left(\frac{n}{2^2}\right) + \frac{3}{2}cn + c\cdot n =  \overset{\text{\textcolor{red}{critical}}}{\overbrace{3^{h}T(1)}} + cn\left(1 + \frac{3}{2} + \left(\frac{3}{2}\right)^2 + ...  + \left(\frac{3}{2}\right)^{h-1} \right) \\
        & h = \log_{2}\left(n\right) \Rightarrow T\left(n\right) = 3^{h}T(1) + c\cdot \textcolor{red}{n}\cdot \left(\left(\frac{3}{\textcolor{red}{2}}\right)^{\log_{2}{n}}\right) / \left(\frac{3}{2} - 1\right) = \theta \left( 3^{\log_{2}(n)} \right) =  \theta \left( n^{\log 3} \right)  
    \end{split}
\end{equation*}
where \(n^{\log 3}  \sim n^{1.58} < n^2 \).
\end{enumerate}



\section{Master Theorem, one Theorem to bound them all. }
As you might already notice, the same pattern has been used to bound both algorithms. The master theorem is the result of the recursive expansion. it classifies recursive functions at the form of \(T\left(n\right) = a\cdot T\left( \frac{n}{b} \right) + f\left(n\right) \), for positive function \(f : \mathbb{N} \rightarrow \mathbb{R}^{+} \).       

\begin{defbox}{Master Theorem, simple version.} First, Consider the case that \(f = n^c\). Let \( a \ge 1, b > 1\) and \( c \ge 0 \). then: 
\begin{enumerate}
    \item if \(\frac{a}{b^c} < 1 \) then \( T\left(n\right) = \Theta \left( n^c \right) \) \ \ \ \textbf{(\(f\) wins)}.
    \item if \(\frac{a}{b^c} = 1 \) then \( T\left(n\right) = \Theta \left( n^c \log_{b} \left(n\right) \right) \).
    \item if \(\frac{a}{b^c} > 1 \) then \( T\left(n\right) = \Theta \left( n^{\log_{b} \left(a\right)} \right) \) \ \ \ \textbf{(\(f\) loose)}.
  \end{enumerate}
\end{defbox}

\paragraph{Example.}  \( T\left(n\right)  =4T\left(\frac{n}{2}\right)+d\cdot n \Rightarrow
T\left(n\right) = \Theta\left(n^2\right)\) according to case (3). And \(T\left(n\right) & = 3T\left(\frac{n}{2}\right) + d\cdot n \Rightarrow T\left(n\right) = \Theta \left( n^{\log_{2}\left(3\right)}\right)\)
also due to case (3).

\begin{defbox}{Master Theorem, strong version.} 
Now, let's generalize the simple version for arbitrary positive \(f\) and let \( a \ge 1, b > 1\). 

\newcommand{\logab}{\log_{b} \left(a\right)}

\begin{enumerate}
    \item if  \(f\left(n\right) = O \left( n^{\logab - \varepsilon }\right)\) for some \( \varepsilon > 0 \) then \( T\left(n\right) = \theta \left( n^{\logab} \right) \) \ \ \ \textbf{(\(f\) loose)}.
    
    \item if  \(f\left(n\right) = \Theta \left( n^{\logab} \right) \) then \( T\left(n\right) = \Theta \left( n^{\logab}  \log\left(n\right)\right) \)
    
    \item if there exist \(\varepsilon >0 ,c<1\) and \(n_0 \in \mathbb{N} \) such that  \(f\left(n\right) = \Omega \left( n^{\logab + \varepsilon }\right)\) and for every \( n > n_0 \) \(a \cdot f\left( \frac{n}{b} \right) \le c f\left(n\right)\)  then \( T\left(n\right) = \theta \left( f\left(n\right) \right) \) \ \ \ \textbf{(\(f\) wins)}.
    
\end{enumerate}
\end{defbox}
\newcommand{\TT}[2]{#1 T\left(\frac{n}{#2}\right)}

\paragraph{Examples}
\begin{enumerate}
    \item \( T\left(n\right) =  T\left(\frac{2n}{3}\right) + 1 \rightarrow f\left(n\right) = 1 =\Theta \left( n^{\log_{\frac{3}{2}} \left(1\right)}\right)\) matches the second case. i.e  \( T\left(n\right) = \Theta \left( n^{\log_{\frac{3}{2}} \left(1\right)}\log n \right)\).
    
    \item \( T\left(n\right) = \TT{3}{4} + n\log n \rightarrow f\left(n\right) = \Omega\left( n^{\log_{4}\left(3\right) + \varepsilon}  \right) \) and notice that \( f\left( a\frac{n}{b}\right) = \frac{3n}{4}\log\left(\frac{3n}{4}\right)\) . Thus, it's matching to the third case. \(\Rightarrow T\left(n\right) = \Theta\left(n\log n\right)\).
    
    \item \(T\left(n\right) = 3T\left( n^{\frac{1}{3}}\right) + \log\log n\). let \( m = \log n \Rightarrow T\left( n\right) = T \left(2^m \right) = 3T\left(2^{\frac{m}{3}} \right) + \log m\). denote by \(S = S\left(m\right) = T\left(2^m\right) \rightarrow S\left(m\right) = 3T\left(2^{\frac{m}{3}} \right) + \log m = 3S\left(\frac{m}{3} \right) + \log m\). And by the fact that \(\log m = O\left(m^{\log_{3}\left(3\right)-\varepsilon} \right) \rightarrow T\left(n\right) = T\left(2^m\right) = S\left(m\right) = \Theta\left(m\right) = \Theta\left( \log(n)\right) \).  
\end{enumerate}


\section{Recursive trees.}
There are still cases which aren't treated by the \textit{Master Theorem}. For example consider the function \(T\left(n\right) = 2T\left(\frac{n}{2}\right) + n\log n \). Note, that \(f = \Omega\left( n^{\log_{b}(a)} \right) = \Omega\left(n\right)\). Yet for every \( \varepsilon > 0 \Rightarrow f = n\log n = O\left( n^{1+\varepsilon} \right) \) therefore the third case  doesn't hold. How can such cases still be analyzed? 

\paragraph{Recursive trees Recipe}
    \begin{enumerate}
        \item draw the computation tree, and calculate it's height. in our case, \( h = \log n \).
        \item calculate the work which done over node at the \(k\)-th level, and the number of nodes in each level. in our case, there are \(2^k\) nodes and over each we perform \(f(n) = \frac{n}{2^k} \log\left( \frac{n}{2^k}\right)\) operations. 
        \item sum up the work of the \(k\)-th level.
        \item finally, the total time is the summation over all the \( k \in [h]\) levels. 
    \end{enumerate}
applying the above, yields 
\begin{equation*} 
\begin{split} 
T\left(n\right) & =  \sum_{k=1}^{\log{n}}{n\cdot\log \left( \frac{n}{2^k}\right)} = n\sum_{k=1}^{\log{n}}{ \left( \log n - \log 2^k \right) } 
  = n\sum_{k=1}^{\log{n}}{ \left( \log n - k \right) } = \\ & = \Theta \left( n \log^2 \left(n\right)  \right) 
\end{split}
\end{equation*}
za
\pzaaragraph{Example.} Consider merge sort variation such that instead of splitting the array into two equals parts it's split them into different size arrays. The first one contains \( \frac{n}{10} \) elements while second contains the others \( \frac{9n}{10}\) elements.

\begin{algbox}{non-equal-merge alg.}
\begin{algorithm}[H]
\SetAlgoLined
\KwResult{returns the sorted permutation of \(x_1 ... x_n \in \mathbb{R}^n \)  }
 \ \\ 
 \If{ \(n \le 10 \) }
    { return bubble-sort \( (x_1 ... x_n)\) } 
 \ \\ 
 
 \Else {
 define \(S_{l} \leftarrow x_1 ... x_{\frac{n}{10}-2}, x_{\frac{n}{10}-1} \) \\
 define \(S_{r} \leftarrow   x_{\frac{n}{10}},x_{\frac{n}{10}+1} ...,x_n \) \\
 \ \\ 
 \( R_l \leftarrow \) non-equal-merge\( \left( S_l \right) \) \\ 
 \( R_r \leftarrow \) non-equal-merge\( \left( S_r \right) \) \\
 \ \\
 return Merge(\(R_l, R_r\))
  
 }
\end{algorithm}
\end{algbox}
Note, that the master theorem achieves an upper bound, 
\begin{equation*}
    \begin{split}
        T\left(n\right) & = n +  T\left(\frac{n}{10}\right) + T\left(\frac{9n}{10}\right) \le n +  2 T\left(\frac{9n}{10}\right) \Rightarrow T\left(n\right) = O \left( n^{\log_{\frac{10}{9}}\left(2\right)} \right) \sim O \left( n^{ 6 } \right)  
    \end{split}
\end{equation*}
Yet, that bound is far from been tight. Let's try to count the operations for each node. Let's try another direction. 
\paragraph{Claim.} Let \(n_i\) be the size of the subset which is processed at the \(i\)-th node. Then for every \(k\):
\begin{equation*}
    \sum_{i \in \text{k level}}{n_i} \le n
\end{equation*}
 \textbf{Proof}. Assuming otherwise implies that there exist index \(j\) such that \(x_j\) appear in at least two different nodes in the same level, denote them by \(u,v\). As they both are in the same level, non of them can be ancestor of the other. denote by \(m \in \mathbb{N}\) the input size of the sub array which is processed by the the lowest common ancestor of \(u\) and \(v\), and by \(j^\prime \in [m]\) the position of \(x_j\) in that sub array. By the definition of the algorithm it steams that \(j^\prime < \frac{m}{10} \) and \(j^\prime \ge \frac{m}{10}\). contradiction.  


 The height of the tree is bounded by \( \log_{\frac{9}{10}} \left(n\right) \). Therefore the total work equals \( \Theta \left( n\log n \right) \).    

Thus, the total running time equals to:
\begin{equation*}
    T(n) = \sum_{k \in \text{levels}}{\sum_{i \in \text{k level}}{f\left(n_i\right)}} = \sum_{k \in \text{levels}}{\sum_{i \in \text{k level}}{n_i}} \le n\log n  
\end{equation*}



\section{Appendix. Recursive Functions In Computer Science. (Optional)}

\ctt{The current section repeats over part of the content above as it was designed to be self-contained. Also, notice that this part is considered as optional material and you are not required to remember the following algorithms for the final exam. Its primary goal is to expose you to "strange" running times. }


\paragraph{Leading Example. numbers multiplication}
Let \(x,y\) be an \(n\)'th digits numbers over \( \mathbb{F}^{n}_{2} \). It's known that summing such a pair requires a linear number of operations. Write an algorithm that calculates the multiplication \(x\cdot y\). 

\paragraph{Example. Long multiplication.} 
To understand the real power of the dividing and conquer method, let's first examine the known solution from elementary school.  In that technics, we calculate the power order and the value of the digit separately and sum up the results at the end. Formally: \(x \leftarrow \sum_{i=0}^{n}{x_{i}2^{i}}\) Thus, \[ x\cdot y =\left( \sum_{i=0}^{n}{x_{i}2^{i}} \right) \left( \sum_{i=0}^{n}{y_{i}2^{i}} \right) =  \sum_{i,j \in [n]\times[n] }{ x_{i}y_{j}2^{i+j} }\] the above is a sum up over \(n^2\) numbers, each at length \(n\) and therefore the total running time of the algorithm is bounded by \( \theta(n^3) \). \ctt{ notice that adding \(1\) to \(1111111111...1\) requires \(O(n)\) }.

\paragraph{Example. Recursive Approach.} We could split \(x\) into the pair \(x_{l}, x_{r}\) such that \(x = x_{l} + 2^{\frac{n}{2}}x_{r} \). Then the multiplication of two \(n\)-long numbers will be reduced to sum up over multiplication of a quartet. Each at length \(\frac{n}{2}\). Thus, the running time is given by \begin{equation*}
    \begin{split}
 x\cdot y & = \left(x_{l} + 2^{\frac{n}{2}}x_{r}\right)\left(y_{l} + 2^{\frac{n}{2}}y_{r}\right) = x_{l}y_{l} + 2^{\frac{n}{2}} \left( x_{l}y_{r} + x_{r}y_{l} \right) + 2^{n}x_{r}y_{r} \\ &  \Rightarrow T\left(n\right)  =4T\left(\frac{n}{2}\right)+c\cdot n=4\cdot4T\left(\frac{n}{4}\right)+4c\cdot\frac{n}{2}+c\cdot n = ... = \\ & c\cdot n\left(1+\frac{4}{2}+\left(\frac{4}{2}\right)^{2}...+\left(\frac{4}{2}\right)^{h-1}\right) + 4^{h}T(1) = n^{2} + c\cdot n\cdot\frac{2^{h}-1}{2-1}
    \end{split}
\end{equation*}
We will call the number of iteration till the stopping condition the recursion height, and we will denote it by \(h\) . What should be the recursion height? \( 2^{h} = n \Rightarrow h =\log\left(n\right) \). So in total we get that multiplication could be achieved by performs \( \Theta\left(n^2\right)\) operations. 
\paragraph{Example. Karatsuba algorithm.} Many years it's was believed that multiplication can't done by less then \( \Omega \left( n^2 \right) \) time; until Karatsuba found the following algorithm. denote by \begin{equation*}
z_0, z_1, z_2 \leftarrow x_{l}y_{r}, x_{l}y_{r} + x_{r}y_{l}, x_{r}y_{r}
\end{equation*}Notice that \( z_1 = \left(x_{l}+x_{r}\right)\left(y_{l}+y_{r}\right) - z_{0} -z_{1} \). summarize the above yields the following pseudo code. 

\begin{algbox}{Karatsuba alg.}
\begin{algorithm}[H]
\SetAlgoLined
\KwResult{returns the multiplication \(x\cdot y\) where \(x,y \in \mathbb{F}^{n}_{2}\) }
 \ \\ 
 \If{ \(x,y \in \mathbb{F}_{2}\) }
    { return \(x \cdot y\) } 
 \ \\ 
 
 \Else {
 define \(x_{l} , x_{r} \leftarrow x \) and \(y_{l} , y_{r} \leftarrow x \) \ \ \ \ \ // \( O \left(n\right) \). \\ 
 \ \\ 
 calculate \(z_0 \leftarrow \text{Karatsuba}\left(x_{l},y_{l}\right)\) \\
 \ \ \ \ \ \ \ \ \ \ \ \ \(z_2 \leftarrow \text{Karatsuba}\left(x_{r},y_{r}\right)\) \\ 
 \ \ \ \ \ \ \ \ \ \ \ \ \(z_1 \leftarrow \text{Karatsuba}\left(x_{r} + x_{l} ,y_{l} + y_{r} \right) - z_0 - z_2 \) \\ 
 \ \\
 return \(z_0 + 2^{\frac{n}{2}}z_1 + 2^{n}z_2\) \ \ \ \ \  // \( O \left(n\right) \). 
 }
\end{algorithm}
\end{algbox}
Let's analyze the running time of the algorithm above, assume that \(n = 2^{m}\) and then the recursive relation is 
\begin{equation*}
    \begin{split}
        T\left(n\right) & = 3T\left(\frac{n}{2}\right) + c\cdot n = 3^2 T\left(\frac{n}{2^2}\right) + \frac{3}{2}cn + c\cdot n = cn\left(1 + \frac{3}{2} + \left(\frac{3}{2}\right)^2 + ...  + \left(\frac{3}{2}\right)^{h-1} \right) + ) + 3^{h}T(1) \\
        & h = \log_{2}\left(n\right) \Rightarrow T\left(n\right) = n^{\log_{2}{3}} +  c\cdot \textcolor{red}{n}\cdot \left(\left(\frac{3}{\textcolor{red}{2}}\right)^{\log_{2}{n}}\right) / \left(\frac{3}{2} - 1\right) = \theta \left( 3^{\log_{2}(n)} \right) =  \theta \left( n^{\log 3} \right)  
    \end{split}
\end{equation*}
where \(n^{\log 3}  \sim n^{1.58} < n^2 \).



\section{Heaps - Recitation 4} 
\author{Correctness, Loop Invariant And Heaps.}


%\begin{paragraph}
  Apart from quantifying the resource requirement of our algorithms, we are also interested in proving that they indeed work. In this Recitation, we will demonstrate how to prove correctness via the notation of loop invariant. In addition, we will present the first (non-trivial) data structure in the course and prove that it allows us to compute the maximum efficiently.

%\end{paragraph}


\subsection*{Correctness And Loop Invariant.}
In this course, any algorithm is defined relative to a task/problem/function, And it will be said correctly if, for any input, it computes desirable output. For example, suppose that our task is to extract the maximum element from a given array. 
So the input space is all the arrays of numbers, and proving that a given algorithm is correct requires proving that the algorithm's output is the maximal number for an arbitrary array. Formally:  
\begin{defbox}{Correctness.}
We will say that an algorithm \( \mathcal{A}\) (an ordered set of operations) computes \( f:D_1 \rightarrow D_2 \) if for every \(x \in D_1 \Rightarrow f(x) = \mathcal{A}(x)\). Sometimes when it's obvious what is the goal function \(f\), we will abbreviate and say that \( \mathcal{A}\) is correct.       
\end{defbox}
\paragraph{}
Other functions \(f\) might be including any computation task: file saving, summing numbers, posting a message in the forum, etc. Let's dive back into the maximum extraction problem and see how one should prove correctness in practice.     
\paragraph{Task: Maximum Finding.} \textit{Given $n\in \mathbb{N}$ numbers $a_1, a_2, \cdots a_n \in \mathbb{R}$ write an Algorithm that returns their maximum.} 

Consider the following suggestion. How would you prove it correct?  
\begin{algbox}{Maximum finding.}
\begin{algorithm*}[H]
% \SetAlgoLined
 \SetKwInOut{Input}{input}
 \Input{ Array  $ a_1, a_2, .. a_n $  }
 let \(b \leftarrow a_1 \) \\ 
 \For{\(i \in [2, n] \) } { 
        \(b \leftarrow \max \left(b, a_i \right) \)
    } 
 return \( b \) 
 %\caption{maximum alg.}
\end{algorithm*}
\end{algbox}
Usually, it will be convenient to divide the algorithms into subsections and then characterize and prove their correctness separately. One primary technique is using the notation of Loop Invariant. Loop Invariant is a property that is characteristic of a loop segment code  and satisfies the following conditions:
\begin{defbox}{Loop Invariant.} 
\begin{enumerate}
  \item Initialization. The property holds (even) before the first iteration of the loop.    
    \item Conservation. As long as one performs the loop iterations, the property still holds.
    \item Termination. Exiting from the loop carrying information.
\end{enumerate}
\end{defbox}

Let's denote by $b^{(i)}$ the value of $b$ at line number $2$ at the $i$th iteration for $i\ge2$ and define $b^{(1)}$ to be its value in its initialization. What is the Loop Invariant here? \textbf{Claim.} \textit{"at the \(i\)-th iteration, $b^{(i)} = \max{ \{ a_1 ... a_{i} \} } $"}. 
\paragraph{Proof.} Initialization, clearly, $ b^{(1)} = a_{1} = \max{ \{ a_1 \} } $. Conservation, by induction, we have the base case from the initialization part, assume the correctness of the claim for any $i^\prime < i$, and consider the $i$th iteration (of course, assume that $i<n$). Then:  
\begin{equation*}
  \begin{split}
b^{(i)} = \max{ \{ b^{(i-1)}, a_{i} \} } = \max{ \{ \max{ \{ a_1, .. a_{i-2}, a_{i-1} \} }, a_{i} \} } = \max{ \{  a_{1}, .. a_{i} \} }
  \end{split}
\end{equation*} 
And that completes the Conservation part. Termination, followed by the conservation, at the $n$ iteration, $b^{(i)}$ is seted to $\max{ \{ a_1 ,a_2 .. a_n  \}}$. 

\paragraph{Task: Element finding.}  \textit{Given $n\in \mathbb{N}$ numbers $a_1, a_2, \cdots a_n \in \mathbb{R}$ and additional number $x \in \mathbb{R}$ write an Algorithm that returns $i$ s.t $a_{i} = x$ if there exists such $i$ and} False \textit{otherwise.} 

\begin{algbox}{Element finding.}
\begin{algorithm}[H]
\SetKwInOut{Input}{input}
 \Input{ Array  $ a_1, a_2, .. a_n $  }
% \SetAlgoLined
 \For{ \(i \in [n] \) } { 
	\If { \(a_{i} = x\) }{
	  return \(i, a_{i}\)
        }
    } 
    return False 
\end{algorithm}
\end{algbox}
\paragraph{Correctness Proof.} First, let's prove the following loop invariant. 
\subparagraph{Claim} \textit{Suppose that the running of the algorithm reached the i-th iteration, then $x \notin \{ a_{1} .. a_{i-1} \}$.} 
\textbf{Proof.} Initialization, for $i=1$ the claim is trivial. Let's use that as the induction base case for proving Conservation. Assume the correctness of the claim for any $i^{\prime} < i$. And consider the $i$th iteration. By the induction assumption, we have that $x \notin \{a_1 .. a_{i-2} \} $, and by the fact that we reached the $i$th iteration, we have that in the $i-1$ iteration, at the line (2) the conditional weren't satisfied (otherwise, the function would return at line (3) namely $x \neq a_{i-1}$. Hence, it follows that $ x \notin \{ a_1, a_2 .. a_{i-2}, a_{i-1} \} $.     
  \subparagraph{} Separate to cases. First, consider the case that given the input $a_1 .. a_n$, the algorithm return False. In this case, we have by the termination property that $x \notin \{ a_1 .. a_n \} $. Now, Suppose that the algorithm returns the pair $\left( i, x \right)$, then it means that the conditional at the line (2) was satisfied at the $i$th iteration. So, indeed $a_{i} = x$, and the algorithm returns the expected output.        


  \newpage
\subsection*{Heaps.}
  \paragraph{Task: The Superpharm Problem. (Motivation for Heaps) }\textit{You are requested to maintain a pharmacy line. In each turn, you get one of the following queries, either a new customer enters the shop, or the pharmacist requests the next person to stand in front. In addition, different customers have different priorities, So you are asked to guarantee that in each turn, the person with the height priority will be at the front.}

Before we consider a sophisticated solution, What is the running time for the naive solution? (maintaining the line as a linear array) ($\sim O\left( n^2 \right)$).

Heaps are structures that enable computing the maximum efficiency in addition to supporting adding and removing elements.
We have seen in the Lecture that no Algorithm can compute the $\max$ function with less than $n-1$ comparisons. So our solution above is indeed the best we could expect for. The same is true for the element-finding problem. Yet, we saw that if we are interested in storing the numbers, then, by keeping them according to sorted order, we could compute each query in logarithmic time via binary search. That raises the question, is it possible to have a similar result regarding the max problem?

\begin{defbox}{Heap}
  Let $n \in \mathbb{N}$ and consider the sequence $H = H_{1}, H_{2} \cdots H_{n} \in \mathbb{R} \left( * \right)$. we will say that $H$ is a Heap if for every $i \in [n]$ we have that: $H_{i} \le H_{2i}, H_{2i + 1}$ when we think of the value at indices greater than $n$ as $H_{i>n} = -\infty$. 
  \begin{equation*}
    \begin{split}
      \Leftrightarrow
    \end{split}
  \end{equation*}
  That definition is equivalent to the following recursive definition: Consider a binary tree in that we associate a number for each node. Then, we will say that this binary tree is a heap if the root's value is lower than its sons' values, and each subtree defined by its children is also a heap. 
\end{defbox}


    % Drawing a graph using the PG 3.0 graphdrawing library
% Author: Mark Wibrow
%\documentclass[tikz,border=10pt]{article}
%\usepackage{tikz}
%\usetikzlibrary{positioning}
%\begin{document}

\tikzset{%
  every neuron/.style={
    circle,
    draw,
    %fill=black,
    radius=0.05mm
  },
  neuron missing/.style={
    draw=none,
    scale=2,
    text height=0.25cm,
    execute at begin node=\color{black}$\vdots$
  },
}

\begin{tikzpicture}
    \node [every  neuron ] (v-1) at ( 0.0 ,-0) { $1$  };\node [every  neuron ] (v-2) at ( 0.0 ,-1) { $2$  };\node [every  neuron ] (v-3) at ( 4.0 ,-1) { $3$  };\node [every  neuron ] (v-4) at ( 0.0 ,-2) { $4$  };\node [every  neuron ] (v-5) at ( 2.0 ,-2) { $5$  };\node [every  neuron ] (v-6) at ( 4.0 ,-2) { $6$  };\node [every  neuron ] (v-7) at ( 6.0 ,-2) { $7$  };\node [every  neuron ] (v-8) at ( 0.0 ,-3) { $8$  };\node [every  neuron ] (v-9) at ( 1.0 ,-3) { $9$  };\node [every  neuron ] (v-10) at ( 2.0 ,-3) { $10$  };\node [every  neuron ] (v-11) at ( 3.0 ,-3) { $11$  };\node [every  neuron ] (v-12) at ( 4.0 ,-3) { $12$  };\node [every  neuron ] (v-13) at ( 5.0 ,-3) { $13$  };\node [every  neuron ] (v-14) at ( 6.0 ,-3) { $14$  };\node [every  neuron ] (v-15) at ( 7.0 ,-3) { $15$  };\draw [->] (v-1) -- (v-2);\draw [->] (v-1) -- (v-3);\draw [->] (v-2) -- (v-4);\draw [->] (v-2) -- (v-5);\draw [->] (v-3) -- (v-6);\draw [->] (v-3) -- (v-7);\draw [->] (v-4) -- (v-8);\draw [->] (v-4) -- (v-9);\draw [->] (v-5) -- (v-10);\draw [->] (v-5) -- (v-11);\draw [->] (v-6) -- (v-12);\draw [->] (v-6) -- (v-13);\draw [->] (v-7) -- (v-14);\draw [->] (v-7) -- (v-15);\end{tikzpicture}
\paragraph{Checking vital signs.}Are the following sequences are heaps? 
\begin{enumerate}
  \item 1,2,3,4,5,6,7,8,9,10 (Y)
  \item 1,1,1,1,1,1,1,1,1,1  (Y)
  \item 1,4,3,2,7,8,9,10     (N)
  \item 1,4,2,5,6,3	     (Y)
\end{enumerate}
\begin{figure}[h]
  \centering
  \begin{subfigure}[b]{0.3\textwidth}
	
    % Drawing a graph using the PG 3.0 graphdrawing library
% Author: Mark Wibrow
%\documentclass[tikz,border=10pt]{article}
%\usepackage{tikz}
%\usetikzlibrary{positioning}
%\begin{document}

\tikzset{%
  every neuron/.style={
    circle,
    draw,
    %fill=black,
    radius=0.05mm
  },
  neuron missing/.style={
    draw=none,
    scale=2,
    text height=0.25cm,
    execute at begin node=\color{black}$\vdots$
  },
}

\begin{tikzpicture}
    \node [every  neuron ] (v-1) at ( 0.0 ,-0) { $1$  };\node [every  neuron ] (v-2) at ( 0.0 ,-1) { $1$  };\node [every  neuron ] (v-3) at ( 2.5 ,-1) { $1$  };\node [every  neuron ] (v-4) at ( 0.0 ,-2) { $1$  };\node [every  neuron ] (v-5) at ( 1.25 ,-2) { $1$  };\node [every  neuron ] (v-6) at ( 2.5 ,-2) { $1$  };\node [every  neuron ] (v-7) at ( 3.75 ,-2) { $1$  };\node [every  neuron ] (v-8) at ( 0.0 ,-3) { $1$  };\node [every  neuron ] (v-9) at ( 0.625 ,-3) { $1$  };\draw [->] (v-1) -- (v-2);\draw [->] (v-1) -- (v-3);\draw [->] (v-2) -- (v-4);\draw [->] (v-2) -- (v-5);\draw [->] (v-3) -- (v-6);\draw [->] (v-3) -- (v-7);\draw [->] (v-4) -- (v-8);\draw [->] (v-4) -- (v-9);\end{tikzpicture}
  \end{subfigure}
\begin{subfigure}[b]{0.3\textwidth}
	\input{tree-3.tex}
  \end{subfigure}
\begin{subfigure}[b]{0.3\textwidth}
	\input{tree-4.tex}
  \end{subfigure}
  \caption{The trees representations of the heaps above. The node which fails to satisfy the Heap inequality is colorized.}
\end{figure}
How much is the cost (running time) to compute the min of $H$? (without changing the heap). ($O\left( 1 \right)$). Assume that option 4 is our Superpharm Line. Let's try to imagine how we should maintain the line. After serving the customer at the top, what can be said on $ \{ H_{2}, H_{3}\}$? or $\{H_{i>3}\}?$ (the second highest value is in $\{H_{2}, H_{3} \}$.)   
\paragraph{Subtask: Extracting Heap's Minimum.} \textit{Let $H$ be an Heap at size $n$, Write algorithm which return $H_1$, erase it and returns $H^\prime$, an Heap which contain all the remain elements.} 
\textbf{Solution:} 

\begin{algbox}{Extract-min.}
\begin{algorithm}[H]
\SetKwInOut{Input}{input}
 \Input{ Heap  $ H_1, H_2, .. H_n $  }
% \SetAlgoLined
ret $\leftarrow H_{1} $ \\
$ H_{1} \leftarrow H_{n} $  \\
$ n \leftarrow n -1 $ \\
Heapify-down$\left( 1 \right)$ \\
return ret  
\end{algorithm}
\end{algbox}



\begin{algbox}{Heapify-down.}
\begin{algorithm}[H]
\SetKwInOut{Input}{input}
 \Input{ Array  $ a_1, a_2, .. a_n $  }
% \SetAlgoLined
next  $\leftarrow i  $ \\
left  $\leftarrow 2i $ \\
right $\rightarrow 2i +1 $ \\ 
\If{ left $ < n \text{ and }  H_{\text{left}} < H_{\text{next}}$ } {
  next $\leftarrow$ left 
}
\If{ right $ < n \text{ and }  H_{\text{right}} < H_{\text{next}}$ } {
  next $\leftarrow$  right
}
\If{ $ i \neq $ next } {
  $ H_{i} \leftrightarrow H_{\text{next}} $ \\ 
  Heapify-down$\left( \text{next}  \right)$
}
\end{algorithm}
\end{algbox}

 

\begin{figure}[h]
  \centering
  \begin{subfigure}[b]{0.23\textwidth}
	\input{tree-41.tex}
  \end{subfigure}
\begin{subfigure}[b]{0.23\textwidth}
	\input{tree-42.tex}
  \end{subfigure}
\begin{subfigure}[b]{0.23\textwidth}
	\input{tree-43.tex}
  \end{subfigure}
\begin{subfigure}[b]{0.23\textwidth}
	
    % Drawing a graph using the PG 3.0 graphdrawing library
% Author: Mark Wibrow
%\documentclass[tikz,border=10pt]{article}
%\usepackage{tikz}
%\usetikzlibrary{positioning}
%\begin{document}

\tikzset{%
  every neuron/.style={
    circle,
    draw,
    %fill=black,
    radius=0.1mm
  },
  neuron missing/.style={
    draw=none,
    scale=2,
    text height=0.25cm,
    execute at begin node=\color{black}$\vdots$
  },
}

\begin{tikzpicture}
    \node [every  neuron ] (v-1) at ( 0.0 ,-0) { $2$  };
  \node [every  neuron ] (v-2) at ( 0.0 ,-1) { $4$  };
\node [every  neuron ] (v-3) at ( 1.5 ,-1) { $2$  };
\node [every  neuron ] (v-4) at ( 0.0 ,-2) { $5$  };
\node [every  neuron ] (v-5) at ( 0.75 ,-2) { $6$  };
\node [every  neuron ] (v-6) at ( 1.5 ,-2) { $3$  };
\draw [->] (v-1) -- (v-2);
\draw [->] (v-1) -- (v-3);
\draw [->] (v-2) -- (v-4);
\draw [->] (v-2) -- (v-5);
\draw [->] (v-3) -- (v-6);
\end{tikzpicture}

  \end{subfigure}
  \caption{Running Example, Extract.} 
\end{figure}


\paragraph{Claim.} Assume that $H$ satisfies the Heap inequality for all the elements except the root. Namely for any $i \neq 1$ we have that $H_{i} \le H_{2i}, H_{2i+1}$. Then applying Heapify-down on $H$ at index $1$ returns a heap.  
\paragraph{Proof.} By induction on the heap size.  
 
\begin{itemize} 
  \item Base, Consider a heap at the size at most three and prove for each by considering each case separately. (lefts as exercise).  
  \item Assumption, assume the correctness of the Claim for any tree that satisfies the heap inequality except the root, at size $n^{\prime} < n$. 
  \item Induction step. Consider a tree at size $n$ which and assume w.l.g (why could we?) that the right child of the root is the minimum between the triple. Then, by the definition of the algorithm, at line 9, the root exchanges its value with its right child. Given that before the swapping, all the elements of the heap, except the root, had satisfied the heap inequality, we have that after the exchange, all the right subtree's elements, except the root of that subtree (the original root's right child) still satisfy the inequality. As the size of the right subtree is at most $n-1$, we could use the assumption and have that after line (10), the right subtree is a heap.  
 
    Now, as the left subtree remains the same (the values of the nodes of the left side didn't change), we have that this subtree is also a heap. So it's left to show that the new tree's root is smaller than its children's. Suppose that is not the case, then it's clear that the root of the right subtree (heap) is smaller than the new root. Combining the fact that its origin must be the right subtree, we have a contradiction to the fact that the original right subtree was a heap (as its root wasn't the minimum element in that subtle).  
 
\end{itemize} 
\paragraph{Question.} How to construct a heap? And how much time will it take?   
\begin{algbox}{Build.}
\begin{algorithm}[H]
  \SetKwInOut{Input}{input}
% \SetAlgoLined
  \Input{ Array $ H = H_{1} .. H_{n} $ }
  $ i \leftarrow \lfloor \frac{1}{2}n  \rfloor $ \\
  \While { $ i > 1 $ }
  { 
    Heapify-down $\left( H, i \right)$ \\ 
    $ i \leftarrow i - 1 $  
  }
return $H_{1} .. H_{n}$
\end{algorithm}
\end{algbox}
We can compute a simple upper bound on the running time of Build as follows. Each call to Heapify costs $O(\log n)$ time, and Build makes $O(n)$ such calls. Thus, the running time is $O(n \log n)$. This upper bound, though correct, is not as tight as it can be.

Let's compute the tight bound. Denote by $U_h$ the subset of vertices in a heap at height $h_{i}$. Also, let $c > 0 $ be the constant quantify the work that is done in each recursive step, then we can express the total cost as being bonded from above by: 

\begin{equation*}
  \begin{split}
    T\left( n \right) & \le \sum_{ h_{i} =1}^{ \log n }{c \cdot \left( \log n -  h_{i} \right)  |U_{h_{i}}|   } 
  \end{split}
\end{equation*}

By counting arguments, we have the inequality $ 2^{\log n - h_{i}}|U_{i}| \le n $ (Proving this argument is left as an exercise). We thus obtain:  

\begin{equation*}
  \begin{split}
    T\left( n \right)  & \le  \sum_{ h_{i} =1}^{ \log n }{c \cdot \left( \log n -  h_{i} \right) \frac{n}{2^{log n - h_{i}} }} = c n \sum_{ j = 1}^{ \log n }{ 2^{-j} \cdot j  }  \\ 
      \le &  c n \sum_{ j = 1}^{ \infty }{ 2^{-j} \cdot j  } 
  \end{split}
\end{equation*}
And by the Ratio test for inifinte serires $ \lim_{j\rightarrow \infty} \frac{(j+1)2^{-j-1}}{j2^{-j}} \rightarrow \frac{1}{2}$ we have that the serire covergence to constant. Hence: $ T\left( n \right) = \Theta\left( n \right) $. 


\paragraph{Heap Sort.}   
Combining the above, we obtain a new sorting algorithm. Given an array, we could first Heapify it (build a heap from it) and then extract the elements by their order. As we saw, the construction requires linear time, and then each extraction costs $\log n$ time. So, the overall cost is $O\left( n\log n  \right)$. Correction follows immediately from Build and Extract correction.  
\begin{algbox}{Heap-Sort.}
\begin{algorithm}[H]
\SetKwInOut{Input}{input}
 \Input{ Array  $ H_1, H_2, .. H_n $  }
% \SetAlgoLined
  $H \leftarrow $ build $ \left( x_1, x_2 .. x_{n}  \right) $ \\ 
  ret $ \leftarrow $ Array $ \{ \} $ \\
  \For {  $ i \in [n] $ } {
  ret$_{i}$ $\leftarrow$ extract $H$
  }
  return ret. 
\end{algorithm}
\end{algbox}

\paragraph{Adding Elements Into The Heap.} (Skip if there is no time).

\begin{algbox}{Heapify-up.}
\begin{algorithm}[H]
  \SetKwInOut{Input}{input}
 \Input{ Heap  $ H_1, H_2, .. H_n $  }
% \SetAlgoLined
parent $\leftarrow \lfloor i/2 \rfloor $ \\
\If{ \text{parent} $  > 0 \text{ and }  H_{\text{parent}} \le H_{i}$ } { 
  $ H_{i} \leftrightarrow H_{\text{parent}} $ \\ 
  Heapify-up$\left( \text{parent}  \right)$
}
\end{algorithm}
\end{algbox}



\begin{algbox}{Insert-key.}
\begin{algorithm}[H]
\SetKwInOut{Input}{input}
 \Input{ Heap  $ H_1, H_2, .. H_n $  }
% \SetAlgoLined
$ H_{n} \leftarrow v $ \\ 
Heapify-up$\left( n \right)$\\
$ n \leftarrow n + 1 $ 
\end{algorithm}
\end{algbox}



\newpage 
\subsection*{Example, Median Heap}

\paragraph{Task:}Write a datastructure that support insertion and deltion at $O\left( \log n \right) $ time and in addition enable to extract the median in $O\left( \log n  \right)$ time. 

\paragraph{Soultlion.} We will define two separate Heaps, the first will be a maximum heap and store the first $ \lfloor n/2 \rfloor $ smallest elements, and the second will be a minimum heap and contain the $ \lceil n/2 \rceil$ greatest elements. So, it's clear that the root of the maximum heap is the median of the elements. Therefore to guarantee correctness, we should maintain the balance between the heap's size. Namely, promising that after each insertion or extraction, the difference between the heap's size is either $0$ or $1$.

\begin{algbox}{Median-Insert-key.}
\begin{algorithm}[H]  
\SetKwInOut{Input}{input} 
\Input{ Array  $ H_1, H_2, .. H_n , v $  } 
\If {$ H_{\max, 1} \le v \le  H_{\min, 1}$ } {
	\If{ size $(H_{\max}) - $ size$(H_{\min}) = 1$} {
       Insert-key ( $H_{min}$, $v$ )
    }
    \Else{
	    Insert-key ( $H_{max}$, $v$ )
    }
}
\Else{
median $\leftarrow$ Median-Extract $H$ \\
\If{ $v < $ median  }{
   Insert-key ( $H_{max}$, $v$ ) \\
   Insert-key ( $H_{min}$, $median$ ) \\
}
\Else{
   Insert-key ( $H_{min}$, $v$ ) \\
   Insert-key ( $H_{max}$, $median$ ) \\
}
}
\end{algorithm}
\end{algbox}

\begin{algbox}{Median-Extract.}
\begin{algorithm}[H]
\SetKwInOut{Input}{input}
 \Input{ Array  $ H_1, H_2, .. H_n $  }
% \SetAlgoLined
median $\leftarrow$ extract $H_{max}$ \\   
\If{ size($H_{min}$) - size($H_{max}$) $> 0$ }{
  temp $ \leftarrow $ extract $H_{\min}$ \\
  Insert-key ( $H_{max}$, temp )    \\
}
return median 
\end{algorithm}
\end{algbox}

\begin{figure}[h]
  \centering
  \begin{subfigure}[b]{0.49\textwidth}
	
    % Drawing a graph using the PG 3.0 graphdrawing library
% Author: Mark Wibrow
%\documentclass[tikz,border=10pt]{article}
%\usepackage{tikz}
%\usetikzlibrary{positioning}
%\begin{document}

\tikzset{%
  every neuron/.style={
    circle,
    draw,
    %fill=black,
    radius=0.1mm
  },
  neuron missing/.style={
    draw=none,
    scale=2,
    text height=0.25cm,
    execute at begin node=\color{black}$\vdots$
  },
}

\begin{tikzpicture}
    \node [every  neuron ] (v-1) at ( 0.0 ,-0) { $143$  };\node [every  neuron ] (v-2) at ( 0.0 ,-1) { $140$  };\node [every  neuron ] (v-3) at ( 4.0 ,-1) { $142$  };\node [every  neuron ] (v-4) at ( 0.0 ,-2) { $124$  };\node [every  neuron ] (v-5) at ( 2.0 ,-2) { $129$  };\node [every  neuron ] (v-6) at ( 4.0 ,-2) { $134$  };\node [every  neuron ] (v-7) at ( 6.0 ,-2) { $131$  };\node [every  neuron ] (v-8) at ( 0.0 ,-3) { $113$  };\node [every  neuron ] (v-9) at ( 1.0 ,-3) { $29$  };\node [every  neuron ] (v-10) at ( 2.0 ,-3) { $16$  };\node [every  neuron ] (v-11) at ( 3.0 ,-3) { $87$  };\node [every  neuron ] (v-12) at ( 4.0 ,-3) { $113$  };\node [every  neuron ] (v-13) at ( 5.0 ,-3) { $24$  };\node [every  neuron ] (v-14) at ( 6.0 ,-3) { $44$  };\node [every  neuron ] (v-15) at ( 7.0 ,-3) { $130$  };\node [every  neuron ] (v-16) at ( 0.0 ,-4) { $77$  };\node [every  neuron ] (v-17) at ( 1.0 ,-4) { $78$  };\draw [->] (v-1) -- (v-2);\draw [->] (v-1) -- (v-3);\draw [->] (v-2) -- (v-4);\draw [->] (v-2) -- (v-5);\draw [->] (v-3) -- (v-6);\draw [->] (v-3) -- (v-7);\draw [->] (v-4) -- (v-8);\draw [->] (v-4) -- (v-9);\draw [->] (v-5) -- (v-10);\draw [->] (v-5) -- (v-11);\draw [->] (v-6) -- (v-12);\draw [->] (v-6) -- (v-13);\draw [->] (v-7) -- (v-14);\draw [->] (v-7) -- (v-15);\draw [->] (v-8) -- (v-16);\draw [->] (v-8) -- (v-17);\end{tikzpicture}
  \end{subfigure}
\begin{subfigure}[b]{0.49\textwidth}
	
    % Drawing a graph using the PG 3.0 graphdrawing library
% Author: Mark Wibrow
%\documentclass[tikz,border=10pt]{article}
%\usepackage{tikz}
%\usetikzlibrary{positioning}
%\begin{document}

\tikzset{%
  every neuron/.style={
    circle,
    draw,
    %fill=black,
    radius=0.1mm
  },
  neuron missing/.style={
    draw=none,
    scale=2,
    text height=0.25cm,
    execute at begin node=\color{black}$\vdots$
  },
}

\begin{tikzpicture}
    \node [every  neuron ] (v-1) at ( 0.0 ,-0) { $185$  };\node [every  neuron ] (v-2) at ( 0.0 ,-1) { $209$  };\node [every  neuron ] (v-3) at ( 4.0 ,-1) { $223$  };\node [every  neuron ] (v-4) at ( 0.0 ,-2) { $229$  };\node [every  neuron ] (v-5) at ( 2.0 ,-2) { $217$  };\node [every  neuron ] (v-6) at ( 4.0 ,-2) { $242$  };\node [every  neuron ] (v-7) at ( 6.0 ,-2) { $280$  };\node [every  neuron ] (v-8) at ( 0.0 ,-3) { $254$  };\node [every  neuron ] (v-9) at ( 1.0 ,-3) { $315$  };\node [every  neuron ] (v-10) at ( 2.0 ,-3) { $218$  };\node [every  neuron ] (v-11) at ( 3.0 ,-3) { $291$  };\node [every  neuron ] (v-12) at ( 4.0 ,-3) { $281$  };\node [every  neuron ] (v-13) at ( 5.0 ,-3) { $296$  };\node [every  neuron ] (v-14) at ( 6.0 ,-3) { $282$  };\node [every  neuron ] (v-15) at ( 7.0 ,-3) { $291$  };\node [every  neuron ] (v-16) at ( 0.0 ,-4) { $330$  };\node [every  neuron ] (v-17) at ( 1.0 ,-4) { $307$  };\draw [->] (v-1) -- (v-2);\draw [->] (v-1) -- (v-3);\draw [->] (v-2) -- (v-4);\draw [->] (v-2) -- (v-5);\draw [->] (v-3) -- (v-6);\draw [->] (v-3) -- (v-7);\draw [->] (v-4) -- (v-8);\draw [->] (v-4) -- (v-9);\draw [->] (v-5) -- (v-10);\draw [->] (v-5) -- (v-11);\draw [->] (v-6) -- (v-12);\draw [->] (v-6) -- (v-13);\draw [->] (v-7) -- (v-14);\draw [->] (v-7) -- (v-15);\draw [->] (v-8) -- (v-16);\draw [->] (v-8) -- (v-17);\end{tikzpicture}
  \end{subfigure}
  \caption{ Example for Median-Heap, the left and right trees are maximum and minimum heaps.  }
\end{figure}

\iffalse 
  \newpage

\section{ Appendix. Exercise from last year }

\paragraph{Question.} Consider the sets $X = \{x_1,x_2 .. x_n\}$, $Y = \{y_1, y_2 .. y_n\}$. Assume that each of the values $x_i,y_i$ is unique. Write an Algorithm which compute the $k$ most small items in $X \oplus Y = \{ x_{i} + y_{j} : x_{i} \in X , y_{j} \in Y  \} $ at $ O \left( n + k\log k  \right) $ time. 

\textbf{Solution.} Notice that If $a \in X$ is greater than $i$ elements of $X$ and $b \in Y$ greater than $j$ elements of $Y$. Then, $a + b$  greater than $i\cdot j$ elements of $X \oplus Y$. Denote by $X^\prime = \{ x^{\prime}_{1} .. x^{\prime}_{n}$ ( $Y^{\prime}$ ) The elements of $X$ in sorted order. So it's clear that if $x_{i}+y_{j} = x^{\prime}_{i^{\prime}} + y^{\prime}_{j^{\prime}}$ is one of the $k$ smallest elements of $X\oplus Y$ then $i^{\prime}j^{\prime} \ge k$. So we will create a heap of elements that respect that inequality and then query that heap.

\begin{algbox}{Heappush.}
\begin{algorithm}[H]
% \SetAlgoLined
$ H_{X} \leftarrow $ build $\left( X \right)$  \\ 
$ H_{Y} \leftarrow $ build $\left( Y \right)$  \\
$ S_{X} \leftarrow $ extract-$k$ $\left( H_{X} \right)$  \\ 
$ S_{Y} \leftarrow $ extract-$k$ $\left( H_{Y} \right)$  \\
$ H_{XY} \leftarrow $ Heap $(\{ \} )$ \\
\For{ $i \in [k]$ } {
  \For { $j \in [k/i]$ } {
  	Heappush( $H_{XY}$, $S_{X,i} + S_{Y, j}$ )    
  }
}
return extract-$k$ ( $H_{XY}$ ) 
\end{algorithm}
\end{algbox}
\fi

\input{../texlib/tail}

